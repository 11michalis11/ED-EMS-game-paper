\subsection{Performance Measures}
One may easily derive the average number of individuals that are at any given state 
using \( pi \). 
The average number of individuals in state \( i \) can be calculated by multiplying 
the number of individuals that are present in state \( i \) with the probability 
of being at that particular state (i.e \(\pi_i (u_i + v_i)\)). 
Using this logic it is possible to calculate any performance measures that are related 
to the mean number of individuals in the system.


Average number of people in the system: 
\begin{equation}
    L = \sum_{i=1}^{|\pi|} \pi_i (u_i + v_i)
\end{equation} 

Average number of people in the service centre: 
\begin{equation}
    L_H = \sum_{i=1}^{|\pi|} \pi_i v_i
\end{equation}

Average number of people in the buffer centre:
\begin{equation}
    L_A = \sum_{i=1}^{|\pi|} \pi_i u_i
\end{equation}

Consequently getting the performance measures that are related to the duration of 
time is not as straightforward. 
Such performance measures are the mean waiting time in the system and the mean time 
blocked in the system. 
Under the scope of this study three approaches have been considered to calculate these 
performance measures; a direct approach, a recursive algorithm and consequently a
closed-form formula.

The research question that needs to be answered here is: ``When a class 1/2 
individuals enters the system, what is the expected time that they will have to 
wait?''. 
In order to formulate the answer to that question one needs to consider all possible 
scenarios of what state the system can be in when an individual arrives. 
Furthermore, different formulas arises for class 1 individuals 
and a different one for class 2 individuals.

\subsubsection{Mean waiting time} 
Upon closer inspection of the recursive formula a more compact formula can arise. 
The equivalent closed-form formula eliminates the need for recursion and thus makes 
the computation of waiting times much more efficient. 
Just like in the recursive part there are two formulas; one for \textit{class 1} 
and one for class 2 individuals. 
The formulas are given by:

\begin{equation} \label{eq:closed_form_waiting_others}
    W^{(1)} = \frac{\sum_{\substack{(u,v) \, \in S_A^{(1)} \\ v \geq C}} 
    \frac{1}{C \mu} \times (v-C+1) \times \pi(u,v)}{\sum_{(u,v) \, 
    \in S_A^{(1)}} \pi(u,v)}
\end{equation}
    
\begin{equation}\label{eq:closed_form_waiting_ambulance}
    W^{(2)} = \frac{\sum_{\substack{(u,v) \, \in S_A^{(2)} \\ min(v,T) \geq C}} 
    \frac{1}{C \mu} \times (\min(v+1,T)-C) \times \pi(u,v)}{\sum_{(u,v) \, 
    \in S_A^{(2)}} \pi(u,v)}
\end{equation}

Note here that the summation, in both equations \ref{eq:closed_form_waiting_others} 
and \ref{eq:closed_form_waiting_ambulance}, goes through all states in the set of 
accepting 
states of either class 1 or class 2 individuals respectively, where a wait 
incurs. 
In equation \ref{eq:closed_form_waiting_others} that includes all states \((u,v)\) 
in the set of accepting states of class 1 individuals such that \( v \geq C\); i.e. 
whenever an arrival occurs and the system is at a state where the number of individuals 
in the system is more than or equal to $C$. 
Consequently, for the states that are included in the summation the expression 
\( v-C+1 \) indicates the amount of people in service one would have to wait for 
upon arrival at the hospital.

Additionally, the minimisation function in equation 
\ref{eq:closed_form_waiting_ambulance} 
ensures that when a class 2 individual arrives at any state 
that is greater than the predetermined threshold, the wait that the individual will 
have to endure remains the same. 
In essence, the expression \(\min(v+1,T) - C\) returns the number of people in line 
in front of a particular individual upon arrival.


\subsubsection{Overall Waiting Time}

Consequently, the overall waiting time should can be estimated by a linear combination 
of the waiting times of class 1 and class 2 individuals. 
The overall waiting time can be then given by the following equation where \(c_1\) 
and \(c_2\) are the coefficients of each individual's type waiting time:

\begin{equation}\label{overall_waiting_time_coeff}
    W = c_1 W^{(1)} + c_2 W^{(2)}
\end{equation}

The two coefficients represent the proportion of individuals of each type that 
traversed through the model. 
Theoretically, getting these percentages should be as simple as looking at the arrival 
rates of each type but in practise if the service centre or the buffer centre 
is full, some individuals may be lost to the system. 
Thus, one should account for the probability that an individual is lost to the system. 
This probability can be easily calculated by using the two sets of accepting states 
\(S_A^{(2)}\) and \(S_A^{(1)}\) defined earlier in equations.
Let us define here the probability, for either class type, that an individual 
is not lost in the system by:

\begin{equation*}
    P(L'_1) = \sum_{(u,v) \, \in S_A^{(1)}} \pi(u,v) \hspace{2cm}
    P(L'_2) = \sum_{(u,v) \, \in S_A^{(2)}} \pi(u,v)
\end{equation*}

Having defined these probabilities one may combine them with the arrival rates of 
each class type in such a way to get the expected proportions of class 1 and 
class 2 individuals in the model. 
Thus, by using these values as the coefficient of equation 
\ref{overall_waiting_time_coeff} 
the resultant equation can be used to get the overall waiting time. 
Note here that the equation below gets the overall waiting time for both the recursive 
and the closed-form formula.

\begin{equation}\label{overall_waiting_time}
    W = \frac{\lambda_1 P(L'_1)}{\lambda_2 P(L'_2) + \lambda_1 P(L'_1)} W^{(1)} + 
    \frac{\lambda_2 P(L'_2)}{\lambda_2 P(L'_2) + \lambda_1 P(L'_1)} W^{(2)}
\end{equation}



\subsubsection{Mean blocking time}
Unlike the waiting time, the blocking time is only calculated for class 2 individuals.  
That is because class 1 individuals cannot be blocked. 
Thus, one only needs to consider the pathway of class 2 individuals to get the 
mean blocking time of the system. 
Blocking occurs at states \((u,v)\) where \(u > 0 \). 
Thus, the set of blocking states can be defined as:

\begin{equation*}
    S_b = \{(u,v) \in S \; | \; u > 0\}
\end{equation*}
 
In order to not consider individuals that will be lost to the system, the set of 
accepting states needs to be taken into account. The set of accepting states is given by:

\begin{equation*}
    S_A^{(2)}=
    \begin{cases}
        \{(u, v) \in S \; | \; u < M \} & \textbf{if } T \leq N\\
        \{(u, v) \in S \; | \; v < N \} & \textbf{otherwise}
    \end{cases}
\end{equation*}

For the waiting time formula,
the mean sojourn time for each state was considered,
ignoring any arrivals. Here, the same approach is used but ignoring only class 2
arrivals. That is because for the waiting time formula, once an individual enters 
the service centre (i.e. starts waiting) any individual arriving after them will 
not affect their
pathway. That is not the case for blocking time. When a class 2 individual is 
blocked, 
any class 1 individual that arrives will cause the blocked individual to remain 
blocked for more time. Therefore, class 1 arrivals are considered here:

\begin{equation}\label{eq:time_in_state_blocking_time}
    c(u,v) = 
    \begin{cases}
        \frac{1}{\min(v,C) \mu}, & \text{if } v = C\\
        \frac{1}{\min(v,C) \mu + \lambda_1}, & \text{otherwise}
    \end{cases}
\end{equation}
 
In equation \ref{eq:time_in_state_blocking_time}, both service completions and 
class 1 arrivals are considered. 
Thus, from a blocked individual's perspective whenever the system moves from one 
state \((u,v)\)
to another state it can either:

\begin{itemize}
    \item be because of a service being completed: we will denote the probability 
    of this happening by \(p_s(u,v)\). 
    \item be because of an arrival of an individual of class 1: denoting such 
    probability by \(p_o(u,v)\).
\end{itemize}
The probabilities are given by:

\begin{equation*}
    p_s(u,v) = \frac{\min(v,C)\mu}{\lambda_1 + \min(v,C)\mu}, \qquad
    p_o(u,v) = \frac{\lambda_1}{\lambda_1 + \min(v,C)\mu}
\end{equation*}


Having defined \(c(u,v)\) and \(S_b\) a formula for the blocking time that is
expected to occur at each state can be given by:

\begin{equation}\label{eq:blocking-time-at-each-state}
    b(u,v) = 
    \begin{cases} 
        0, & \textbf{if } (u,v) \notin S_b \\
        c(u,v) + b(u - 1, v), & \textbf{if } v = N = T\\
        c(u,v) + b(u, v-1), & \textbf{if } v = N \neq T \\
        c(u,v) + p_s(u,v) b(u-1, v) + p_o(u,v) b(u, v+1), & \textbf{if } u > 0 
        \textbf{ and } v = T \\
        c(u,v) + p_s(u,v) b(u, v-1) + p_o(u,v) b(u, v+1), & \textbf{otherwise} \\
    \end{cases}
\end{equation}

Equation 
(\ref{eq:blocking-time-at-each-state}) will not be solved recursively. 
A direct approach will be used to solve this equation here. 
By enumerating all equations of (\ref{eq:blocking-time-at-each-state}) for all 
states \((u,v)\) that belong in \(S_b\) 
a system of linear equations arises where the unknown variables are all the \(b(u,v)\)
terms.
For instance, let us consider a Markov model where \(C=2, T=3, N=6, M=2\). 
The Markov model is shown in Figure \ref{fig:example-algeb-blocking}
and the equivalent equations are 
(\ref{eq:first_eq_of_blocking_example})-(\ref{eq:last_eq_of_blocking_example}).
The equations considered here are only the ones that correspond to the blocking 
states.

\begin{multicols*}{2}
    \begin{figure}[H]
        \scalebox{0.50}{\begin{tikzpicture}[-, node distance = 1cm, auto]
\node[state] (u0v0) {(0,0)};
\node[state, right=of u0v0] (u0v1) {(0,1)};
\draw[->](u0v0) edge[bend left] node {\( \Lambda \)} (u0v1);
\draw[->](u0v1) edge[bend left] node {\(\mu \)} (u0v0);
\node[state, right=of u0v1] (u0v2) {(0,2)};
\draw[->](u0v1) edge[bend left] node {\( \Lambda \)} (u0v2);
\draw[->](u0v2) edge[bend left] node {\(2\mu \)} (u0v1);
\node[state, below=of u0v2] (u1v2) {(1,2)};
\draw[->](u0v2) edge[bend left] node {\( \lambda_2 \)} (u1v2);
\draw[->](u1v2) edge[bend left] node {\(2\mu \)} (u0v2);
\node[state, below=of u1v2] (u2v2) {(2,2)};
\draw[->](u1v2) edge[bend left] node {\( \lambda_2 \)} (u2v2);
\draw[->](u2v2) edge[bend left] node {\(2\mu \)} (u1v2);
\node[state, right=of u0v2] (u0v3) {(0,3)};
\draw[->](u0v2) edge[bend left] node {\( \lambda_1 \)} (u0v3);
\draw[->](u0v3) edge[bend left] node {\(2\mu \)} (u0v2);
\node[state, right=of u1v2] (u1v3) {(1,3)};
\draw[->](u1v2) edge[bend left] node {\( \lambda_1 \)} (u1v3);
\draw[->](u1v3) edge[bend left] node {\(2\mu \)} (u1v2);
\draw[->](u0v3) edge node {\( \lambda_2 \)} (u1v3);
\node[state, right=of u2v2] (u2v3) {(2,3)};
\draw[->](u2v2) edge[bend left] node {\( \lambda_1 \)} (u2v3);
\draw[->](u2v3) edge[bend left] node {\(2\mu \)} (u2v2);
\draw[->](u1v3) edge node {\( \lambda_2 \)} (u2v3);
\node[state, right=of u0v3] (u0v4) {(0,4)};
\draw[->](u0v3) edge[bend left] node {\( \lambda_1 \)} (u0v4);
\draw[->](u0v4) edge[bend left] node {\(2\mu \)} (u0v3);
\node[state, right=of u1v3] (u1v4) {(1,4)};
\draw[->](u1v3) edge[bend left] node {\( \lambda_1 \)} (u1v4);
\draw[->](u1v4) edge[bend left] node {\(2\mu \)} (u1v3);
\draw[->](u0v4) edge node {\( \lambda_2 \)} (u1v4);
\node[state, right=of u2v3] (u2v4) {(2,4)};
\draw[->](u2v3) edge[bend left] node {\( \lambda_1 \)} (u2v4);
\draw[->](u2v4) edge[bend left] node {\(2\mu \)} (u2v3);
\draw[->](u1v4) edge node {\( \lambda_2 \)} (u2v4);
\end{tikzpicture}}
        \caption{Example of Markov chain}
        \label{fig:example-algeb-blocking}
    \end{figure}
    \columnbreak
    \begin{align}
        b(1,2) &= c(1,2) + p_o b(1,3) \label{eq:first_eq_of_blocking_example} \\
        b(1,3) &= c(1,3) + p_s b(1,2) + p_o b(1,4) \\
        b(1,4) &= c(1,4) + b(1,3) \\
        b(2,2) &= c(2,2) + p_s b(1,2) + p_o b(2,3) \\
        b(2,3) &= c(2,3) + p_s b(2,2) + p_o b(1,4) \\
        b(2,4) &= c(2,4) + b(2,3)\label{eq:last_eq_of_blocking_example}
    \end{align}
\end{multicols*}

Additionally, the above equations can be transformed into a linear system of the 
form \(Zx=y\) where:

\begin{equation}\label{eq:example-algebaric-approach-blocking-time}
    Z=
    \begin{pmatrix}
        -1 & p_o & 0 & 0 & 0 & 0 \\ %(1,2)
        p_s & -1 & p_o & 0 & 0 & 0 \\ %(1,3)
        0 & 1 & -1 & 0 & 0 & 0 \\ %(1,4)
        p_s & 0 & 0 & -1 & p_o & 0\\ %(2,2)
        0 & 0 & 0 & p_s & -1 & p_o \\ %(2,3)
        0 & 0 & 0 & 0 & 1 & -1 \\ %(2,4)
    \end{pmatrix},
    x=
    \begin{pmatrix}
        b(1,2) \\
        b(1,3) \\
        b(1,4) \\
        b(2,2) \\
        b(2,3) \\
        b(2,4) \\
    \end{pmatrix}, 
    y=
    \begin{pmatrix}
        -c(1,2) \\
        -c(1,3) \\
        -c(1,4) \\
        -c(2,2) \\
        -c(2,3) \\
        -c(2,4) \\
    \end{pmatrix}
\end{equation}

A more generalised form of the equations in 
(\ref{eq:example-algebaric-approach-blocking-time})
can thus be given for any value of \(C,T,N,M\) by:

\begin{align}
    b(1,T) =& c(1, T) + p_o b(1, T + 1) \label{eq:first_eq_of_blocking_general}\\
    b(1,T + 1) =& c(1, T + 1) + p_s(1, T) + p_o b(1, T + 1) \\
    b(1,T + 2) =& c(1, T + 2) + p_s(1, T + 1) + p_o b(1, T + 3) \\
    & \vdots \nonumber \\
    b(1, N) =& c(1, N) + b(1, N - 1) \\
    b(2, T) =& c(2, T) + p_s b(1, T) + p_o b(2, T + 1) \\
    b(2, T + 1) =& c(2, T + 1) + p_s b(2, T) + p_o b(2, T + 2) \\
    & \vdots \nonumber \\
    b(M, T) =& c(M, T) + b(M, T-1) \label{eq:last_eq_of_blocking_general}
\end{align}

The equivalent matrix form of the linear system of equations 
(\ref{eq:first_eq_of_blocking_general}) - (\ref{eq:last_eq_of_blocking_general})
is given by \(Zx=y\), where:
\begin{equation}\label{eq:general-algebaric-approach-blocking-time}
    \scalebox{0.9}{
        \(
        Z = 
        \begin{pmatrix}
            -1 & p_o & 0 & \dots & 0 & 0 & 0 & 0 & 0 & \dots & 0 & 0 \\ %(1,T)
            p_s & -1 & p_o & \dots & 0 & 0 & 0 & 0 & 0 & \dots & 0 & 0 \\ %(1,T+1)
            0 & p_s & -1 & \dots & 0 & 0 & 0 & 0 & 0 & \dots & 0 & 0 \\ %(1,T+2)
            \vdots & \vdots & \vdots & \ddots & \vdots & \vdots & \vdots & \vdots & 
            \vdots & \ddots & \vdots & \vdots \\ 
            0 & 0 & 0 & \dots & 1 & -1 & 0 & 0 & 0 & \dots & 0 & 0 \\ %(1,N)
            p_s & 0 & 0 & \dots & 0 & 0 & -1 & p_o & 0 & \dots & 0 & 0 \\ %(2,T)
            0 & 0 & 0 & \dots & 0 & 0 & p_s & -1 & p_o & \dots & 0 & 0 \\ %(2,T+1)
            \vdots & \vdots & \vdots & \ddots & \vdots & \vdots & \vdots & \vdots & 
            \vdots & \ddots & \vdots & \vdots \\ 
            0 & 0 & 0 & \dots & 0 & 0 & 0 & 0 & 0 & \dots & 1 & -1 \\ %(M,T)
        \end{pmatrix},
        x = 
        \begin{pmatrix}
            b(1,T) \\
            b(1,T+1) \\
            b(1,T+2) \\
            \vdots \\
            b(1,N) \\
            b(2,T) \\
            b(2,T+1) \\
            \vdots \\
            b(M,T) \\
        \end{pmatrix}, 
        y= 
        \begin{pmatrix}
            -c(1,T) \\
            -c(1,T+1) \\
            -c(1,T+2) \\
            \vdots \\
            -c(1,N) \\
            -c(2,T) \\
            -c(2,T+1) \\
            \vdots \\
            -c(M,T) \\
        \end{pmatrix}
        \)
    }
\end{equation}

Thus, having calculated the mean blocking time for all blocking states \(b(u,v)\), 
it only remains to put them together in a formula.
The resultant blocking time formula is given by:

\begin{equation}\label{eq:algebraic-blocking-time}
    B = \frac{\sum_{(u,v) \in S_A} \pi_{(u,v)} \; b(u,v)}{\sum_{(u,v) \in S_A} 
    \pi_{(u,v)}}
\end{equation}
