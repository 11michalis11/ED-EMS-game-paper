\section{A queueing model with 2 consecutive buffer centres}

% TODO: If this is to be one paper maybe the first two paragraphs do not need to
% include so much details or even better omit some of the details in the game
% overview section.
In this section, a more in-depth explanation of the queueing model shown in 
figure \ref{fig:diagram_of_queueing_system} will be given.
This is a queuing model that consists of two waiting spaces, one for each type
of patient.

The model consists of two types of individuals; class 1 and class 2.
Class 1 individuals arrive instantly at waiting zone 1 and proceed to wait to
receive their service. 
Class 2 individuals arrive at waiting zone 2 and wait there until they are 
allowed to move to waiting zone 1. 
They are allowed to proceed to the next waiting zone only when the number of 
individuals in the next system (in waiting zone 1 and in service) is less than a 
pre-determined threshold \(T\).
When the threshold is reached, all class 2 individuals that arrive will remain 
\textit{``blocked''} in the buffer centre until the number of people in the 
system is reduced below \(T\). 
This is shown diagrammatically in figure \ref{fig:diagram_of_queueing_system}.
The parameters of the described queueing model are:

\begin{itemize}
    \item $\lambda_1$: The arrival rate of class 1 individuals
    \item $\lambda_2$: The arrival rate of class 2 individuals
    \item $\mu$: The service rate for individuals receiving service in waiting 
    zone 1
    \item $C$: The number of servers
    \item $T$: The threshold where class 2 individuals will start being blocked 
\end{itemize}

In order to model such queueing system a Markov chain model was used.
Markov chains are a special kind of memoryless stochastic systems where the 
outcome of the next iteration depends only on the current state
~\cite{kemeny1976markov}. 
The way such models are formulated is through a directed graph where each vertex
is consider a state of the system and each edge represents the transition rate
between its two linking vertices.
The states of the Markov chain are denoted by \((u,v)\) where:

\begin{itemize}
    \item \(u\) = number of individuals blocked
    \item \(v\) = number of individuals either in waiting zone 1 or in the
    service centre
\end{itemize}


The state space of the Markov chain \(S\) is defined as the union between
\(S_1\) and \(S_2\), where \(S_1\) represents the set of states where the total 
number of individuals is less than \(T\) and \(S_2\) the set of states where the
total number of individuals is greater than \(T\). 

\begin{align}
    S(T) =& S_1(T) \cup S_2(T) \text{ where:} \nonumber \\
    S_1(T) =& \left\{(0, v)\in\mathbb{N}_0^2 \; | \; v < T \right\} 
    \label{eq:state_space} \\
    S_2(T) =& \{(u, v)\in\mathbb{N}_0^2 \; | \; v \geq T \} \nonumber
\end{align}

The transition matrix \(Q\) of the Markov chain consists of the transition rates
between the numerous states of the model. Every entry \( q_{i,j} = 
q_{(u_i, v_i),(u_j, v_j)} \) represents the transition rate from state 
\( i = (u_i, v_i) \) to state \( j = (u_j , v_j) \) for all 
\( (u_i, v_i), (u_j, v_j) \in S \).
The entries of \(Q\) can be calculated using the state-mapping function 
described in \ref{eq:markov_transition_rate}: 

\begin{equation} \label{eq:markov_transition_rate}
    q_{i, j} = 
    \begin{cases}
        \Lambda, & \textbf{if } (u_i, v_i) - (u_j, v_j) = (0,-1) \textbf{ and } 
        v_i < \text{t} \\
        \lambda_1, & \textbf{if } (u_i, v_i) - (u_j, v_j) = (0,-1) 
        \textbf{ and } v_i \geq \text{t} \\
        \lambda_2, & \textbf{if } (u_i, v_i) - (u_j, v_j) = (-1,0) \\
        v_i \mu, & \textbf{if } (u_i, v_i) - (u_j, v_j) = (0,1) \textbf{ and } 
        v_i \leq C \textbf{ or} \\ & \hspace{0.37cm}(u_i, v_i) - (u_j, v_j) = 
        (1,0) \textbf{ and } v_i = T \leq C \\
        C \mu, & \textbf{if } (u_i, v_i) - (u_j, v_j) = (0,1) \textbf{ and } 
        v_i > C 
        \textbf{ or} \\ & \hspace{0.37cm}(u_i, v_i) - (u_j, v_j) = (1,0) 
        \textbf{ and } v_i = T > C\\
        -\sum_{j=1}^{|Q|}{q_{i,j}} & \textbf{if } i = j \\
        0, & \textbf{otherwise}
    \end{cases}
\end{equation}

Note that $\Lambda$ here denotes the overall arrival rate in the model by both 
classes of individuals (i.e. $\Lambda = \lambda_1 + \lambda_2$). 
A visualisation of how the transition rates relate to the states of the model 
can be seen in the general Markov chain model shown in figure 
\ref{fig:general-markov-model}.

\begin{figure}[H]
    \centering
    \scalebox{.8}
    {
        \begin{tikzpicture}[-, node distance = 0.9cm, auto, every node/.style={scale=0.7}]

            % Markov chain variables
            \tikzmath{
                let \initdist = 0.5cm;
                let \altdist = 1.2cm;
                let \minsz = 1.6cm;
            }

            % S_1 and S_2 rectangles
            \tikzmath{
                let \leftOne = -0.8;
                let \rightOne = 2.7;
                let \upOne = 0.8;
                let \downOne = -2.7;
                let \leftTwo = 2.8;
                let \rightTwo = 13;
                let \upTwo = -2.95;
                let \downTwo = -16.4;
            }

            % General case variables
            \tikzmath{
                let \GCsmallx = 8.3;
                let \GCsmally = -9.5;
                let \GCbigx = 4.1;
                let \GCbigy = -11.8;
            }

            % Rectangle for S1
            \draw[ultra thin, dashed] (\leftOne, \downOne) -- (\leftOne, \upOne);
            \draw[ultra thin, dashed] (\leftOne, \upOne) -- (\rightOne, \upOne);
            \draw[ultra thin, dashed] (\rightOne, \upOne) -- node 
            {\Huge{\( \quad S_1 \)}}(\rightOne, \downOne);
            \draw[ultra thin, dashed] (\rightOne, \downOne) -- (\leftOne, \downOne);

            % Rectangle for S2
            \draw[ultra thin, dashed] (\leftTwo, \downTwo) -- node 
            {\Huge{\( S_2 \quad \)}}(\leftTwo, \upTwo);
            \draw[ultra thin, dashed] (\leftTwo, \upTwo) -- (\rightTwo, \upTwo);
            \draw[ultra thin, dashed] (\rightTwo, \upTwo) -- (\rightTwo, \downTwo);
            \draw[ultra thin, dashed] (\rightTwo, \downTwo) -- (\leftTwo, \downTwo);

            % Small square of general case
            \draw [thick] (\GCsmallx, \GCsmally) -- node {} 
            (\GCsmallx + 0.4, \GCsmally);
            \draw [thick] (\GCsmallx + 0.4, \GCsmally) -- node {} 
            (\GCsmallx + 0.4, \GCsmally - 0.4);
            \draw [thick] (\GCsmallx + 0.4, \GCsmally - 0.4) -- node {} 
            (\GCsmallx, \GCsmally - 0.4);
            \draw [thick] (\GCsmallx, \GCsmally - 0.4) -- node {} 
            (\GCsmallx, \GCsmally);


            % Dashed lines to from small square to big one 
            \draw [ultra thin] (\GCsmallx, \GCsmally) -- node {} 
            (\GCbigx, \GCbigy);
            \draw [ultra thin] (\GCsmallx + 0.4, \GCsmally) -- node {} 
            (\GCbigx + 4, \GCbigy);
            \draw [ultra thin] (\GCsmallx, \GCsmally - 0.4) -- node {} (7, \GCbigy);
            \draw [ultra thin] (\GCsmallx + 0.4, \GCsmally - 0.4) -- node {} 
            (\GCbigx + 4, \GCbigy - 4);
            
            % Big Square of general case
            \draw [ultra thick] (\GCbigx, \GCbigy) -- node {} (\GCbigx + 4, \GCbigy);
            \draw [ultra thick] (\GCbigx + 4, \GCbigy) -- node {} 
            (\GCbigx + 4, \GCbigy - 4);
            \draw [ultra thick] (\GCbigx + 4, \GCbigy - 4) -- node {General Case} 
            (\GCbigx, \GCbigy - 4);
            \draw [ultra thick] (\GCbigx, \GCbigy - 4) -- node {} (\GCbigx, \GCbigy);

            % First Line
            \node[state, minimum size=1.5cm] (zero) {(0,0)};
            \node[state, node distance = \initdist, minimum size=\minsz, below right=of zero] 
            (one) {(0,1)};
            \node[draw=none, node distance = \initdist, minimum size=\minsz, below right=of one] 
            (two) {\textbf{\( \ddots \)}};
            \node[state, node distance = \initdist, minimum size=\minsz, below right=of two] 
            (three) {(0,T)};
            \node[state, node distance = \altdist, minimum size=\minsz, right=of three] 
            (four) {(0,T+1)};
            \node[draw=none, node distance = \altdist, minimum size=\minsz, right=of four] 
            (five) {\textbf{\dots}};
            \node[state, minimum size=\minsz, right=of five] (six) {(0,C)};
            \node[draw=none, minimum size=\minsz, right=of six] (seven) {\textbf{\dots}};

            % Second Line
            \node[state, minimum size=\minsz, below=of three] (three_one) {(1,T)};
            \node[state, minimum size=\minsz, below=of four] (four_one) {(1,T+1)};
            \node[draw=none, minimum size=\minsz, below=of five] (five_one) {\textbf{\dots}};
            \node[state, minimum size=\minsz, right=of five_one] (six_one) {(1,C)};
            \node[draw=none, minimum size=\minsz, right=of six_one] (seven_one) {\textbf{\dots}};
            
            % Third Line
            \node[state, minimum size=\minsz, below=of three_one] (three_two) {(2,T)};
            \node[state, minimum size=\minsz, below=of four_one] (four_two) {(2,T+1)};
            \node[draw=none, minimum size=\minsz, below=of five_one] (five_two) 
            {\textbf{\dots}};
            \node[state, minimum size=\minsz, right=of five_two] (six_two) {(2,C)};
            \node[draw=none, minimum size=\minsz, right=of six_two] (seven_two) 
            {\textbf{\dots}};

            % Fourth line
            \node[draw=none, node distance = \altdist, minimum size=\minsz, below=of three_two] 
            (three_three) {\textbf{\vdots}};
            \node[draw=none, node distance = \altdist, minimum size=\minsz, below=of four_two] 
            (four_three) {\textbf{\vdots}};
            \node[draw=none, node distance = 2cm, minimum size=\minsz, below=of five_two] 
            (five_three) {};
            \node[draw=none, node distance = \altdist, minimum size=\minsz, below=of six_two] 
            (six_three) {\textbf{\vdots}};

            % Fifth line
            \node[draw=none, node distance = 0.3cm, minimum size=\minsz, below=of four_three] 
            (general_case_up) {};
            \node[state, node distance = \altdist, minimum size=\minsz, below=of general_case_up] 
            (general_case_mid) {\( (u_i, v_i) \)};

            \node[draw=none, node distance = \altdist, minimum size=\minsz, below=of general_case_mid] 
            (general_case_down) {};
            \node[draw=none, node distance = \altdist, minimum size=\minsz, left=of general_case_mid] 
            (general_case_left) {};
            \node[draw=none, node distance = \altdist, minimum size=\minsz, right=of general_case_mid] 
            (general_case_right) {};

            \draw[every loop]
                % First Horizontal Edges
                (zero) edge[bend left] node {\( \Lambda \)} (one)
                (one) edge[bend left] node {\( \mu \)} (zero)
                (one) edge[bend left] node {\( \Lambda \)} (two)
                (two) edge[bend left] node {\( 2 \mu \)} (one)
                (two) edge[bend left] node {\( \Lambda \)} (three)
                (three) edge[bend left] node {\( T \mu \)} (two)
                (three) edge[bend left] node {\( \lambda_1 \)} (four)
                (four) edge[bend left] node {\( (T+1) \mu \)} (three)
                (four) edge[bend left] node {\( \lambda_1 \)} (five)
                (five) edge[bend left] node {\( (T+2) \mu \)} (four)
                (five) edge[bend left] node {\( \lambda_1 \)} (six)
                (six) edge[bend left] node {\( C\mu \)} (five)
                (six) edge[bend left] node {\( \lambda_1 \)} (seven)
                (seven) edge[bend left] node {\( C\mu \)} (six)

                % Second Horizontal Edges
                (three_one) edge[bend left] node {\( \lambda_1 \)} (four_one)
                (four_one) edge[bend left] node {\( (T+1) \mu \)} (three_one)
                (four_one) edge[bend left] node {\( \lambda_1 \)} (five_one)
                (five_one) edge[bend left] node {\( (T+2) \mu \)} (four_one)
                (five_one) edge[bend left] node {\( \lambda_1 \)} (six_one)
                (six_one) edge[bend left] node {\( C\mu \)} (five_one)
                (six_one) edge[bend left] node {\( \lambda_1 \)} (seven_one)
                (seven_one) edge[bend left] node {\( C\mu \)} (six_one)

                % Third Horizontal Edges
                (three_two) edge[bend left] node {\( \lambda_1 \)} (four_two)
                (four_two) edge[bend left] node [below] {\( (T+1) \mu \)} (three_two)
                (four_two) edge[bend left] node {\( \lambda_1 \)} (five_two)
                (five_two) edge[bend left] node {\( (T+2) \mu \)} (four_two)
                (five_two) edge[bend left] node {\( \lambda_1 \)} (six_two)
                (six_two) edge[bend left] node {\( C\mu \)} (five_two)
                (six_two) edge[bend left] node {\( \lambda_1 \)} (seven_two)
                (seven_two) edge[bend left] node {\( C\mu \)} (six_two)

                % First Vertical Edges
                (three) edge[bend left] node {\( \lambda_2 \)} (three_one)
                (three_one) edge[bend left] node {\( T \mu \)} (three)
                (three_one) edge[bend left] node {\( \lambda_2 \)} (three_two)
                (three_two) edge[bend left] node {\( T\mu \)} (three_one)
                (three_two) edge[bend left] node {\( \lambda_2 \)} (three_three)
                (three_three) edge[bend left] node {\( T\mu \)} (three_two)

                % Second Vertical Edges
                (four) edge node {\( \lambda_2 \)} (four_one)
                (four_one) edge node {\( \lambda_2 \)} (four_two)
                (four_two) edge node {\( \lambda_2 \)} (four_three)

                % Fourth Vertical Edges
                (six) edge node {\( \lambda_2 \)} (six_one)
                (six_one) edge node {\( \lambda_2 \)} (six_two)
                (six_two) edge node {\( \lambda_2 \)} (six_three)

                % General Case
                (general_case_left) edge[bend left] node {\( \lambda_1 \)} (general_case_mid)
                (general_case_mid) edge[bend left] node {\( v_i \mu \)} (general_case_left)
                (general_case_right) edge[bend left] node {\( (v_i +1) \mu \)} (general_case_mid)
                (general_case_mid) edge[bend left] node {\( \lambda_1 \)} (general_case_right)
                % (five_three) edge node {\( \lambda_2 \)} (general_case_mid)
                (general_case_up) edge node {\( \lambda_2 \)} (general_case_mid)
                (general_case_mid) edge node {\( \lambda_2 \)} (general_case_down)
                ;
        \end{tikzpicture}
    }
    \caption{General case of the Markov chain model} 
    \label{fig:general-markov-model}
\end{figure}



In order to acquire an exact solution of the problem a slight adjustment needs to 
be considered. 
The problem defined above assumes no upper boundary to the number of individuals 
that can wait for service or for the ones that are blocked in the buffer centre. 
Therefore, a different state space \( \tilde S \) needs to be constructed where 
\( \tilde S \subseteq S \) and there is a maximum allowed number of people \(N\) 
that can be in the system and a maximum allowed number of people \(M\) that can
be blocked in the buffer centre:

\begin{equation}
    \tilde S = \left\{ (u, v) \in S\;| u \leq M, v\leq N \right\}
\end{equation}


\subsection{Performance Measures}


The transition matrix \( Q \) defined in \ref{eq:markov_transition_rate} can be 
used to get the probability vector \( \pi \).
The vector \( \pi \) is commonly used to study stochastic systems and it's main
purpose is to keep track of the probability of being at any given state of 
the system.
The term \textit{steady state} refers to the instance of the vector \( \pi \) 
where the probabilities of being at any state become stable over time. 
Thus, by considering the steady state vector \( \pi \) the relationship between 
it and \( Q \) is given by:

\[
    \frac{d\pi}{dt} = \pi Q = 0
\]

Using vector \(\pi\) there are numerous performance measures of the model that 
can be calculated. 
The following equations utilise \(\pi\) to get performance measures on the 
average number of people at certain sets of state.

\begin{itemize}
    \item Average number of people in the system: 
        \[L = \sum_{i=1}^{|\pi|} \pi_i (u_i + v_i)\]
    \item Average number of people in the service centre: 
        \[L_H = \sum_{i=1}^{|\pi|} \pi_i v_i\]
    \item Average number of people in waiting zone 2:
        \[L_A = \sum_{i=1}^{|\pi|} \pi_i u_i\] 
\end{itemize}

Consequently, there are some additional performance measures of interest that
are not as straightforward to calculate.
Such performance measures are the mean waiting time in the system (for both 
class 1 and class 2 individuals), the mean time blocked in waiting zone 2 (only 
valid for class 2 individuals) and the proportion of individuals that wait in 
waiting zone 1 within a predefined time target.

\subsubsection{Waiting time} \label{sec:waiting_time}

Waiting time is the amount of time that individuals from either class wait in 
waiting zone 1 so that they can receive their service. 
For a given set of parameters there are three different performance measures 
around the mean waiting time that can be calculated; the mean waiting time of
class 1 individuals, the mean waiting time of class 2 individuals and the 
overall mean waiting time. 

Since some of the individuals can be lost to the model, a new set of states 
needs to be defined; the set of \textit{accepting states}. 
That is the set of states that the model is able to accept a certain type of
individual. 
The set of accepting states for class 1 individuals is defined as:

\begin{equation}\label{eq:accepting_states_class_1}
    S_A^{(1)} = \{(u, v) \in S \; | \; v < N \}
\end{equation}

In essence, for class 1 individuals, this is the set of states that are not on 
the last column of states in the Markov chain.
Equivalently, the set of accepting states for class 2 individuals is defined as:

\begin{equation}\label{eq:accepting_states_class_2}
    S_A^{(2)}=
    \begin{cases}
        \{(u, v) \in S \; | \; u < M \}, & \textbf{if } T \leq N\\
        \{(u, v) \in S \; | \; v < N \}, & \textbf{otherwise}
    \end{cases}
\end{equation}

Note here that if the threshold is less than or equal the total capacity of the
system the set includes all states that are not on the last column of the 
Markov chain.
Otherwise, the set of accepting state is identical to 
(\ref{eq:accepting_states_class_1}). Thus, the expressions for the waiting times 
for class 1 and class 2 individuals are given by:

\begin{equation} \label{eq:closed_form_waiting_class_1}
    W^{(1)} = \frac{\sum_{\substack{(u,v) \, \in S_A^{(1)} \\ v \geq C}} 
    \frac{1}{C \mu} \times (v-C+1) \times \pi(u,v)}{\sum_{(u,v) \, 
    \in S_A^{(1)}} \pi(u,v)}
\end{equation}
    
\begin{equation}\label{eq:closed_form_waiting_class_2}
    W^{(2)} = \frac{\sum_{\substack{(u,v) \, \in S_A^{(2)} \\ min(v,T) \geq C}} 
    \frac{1}{C \mu} \times (\min(v+1,T)-C) \times \pi(u,v)}{\sum_{(u,v) \, 
    \in S_A^{(2)}} \pi(u,v)}
\end{equation}

Consequently, the overall waiting time can be estimated by a linear combination 
of \(W_1\) and \(W_2\). 
Thus, the overall waiting time can calculated by the following equation where 
\(c_1\) and \(c_2\) are the coefficients of the terms:

\begin{equation}\label{eq:overall_waiting_time_coeff}
    W = c_1 W^{(1)} + c_2 W^{(2)}
\end{equation}

The two coefficients represent the proportion of individuals of each type that 
did not get lost and traversed through the model.
Thus, one should account for the probability that an individual is lost to the 
system. 
This probability can be easily calculated by using the two sets of accepting 
states \(S_A^{(2)}\) and \(S_A^{(1)}\) defined in equations 
(\ref{eq:accepting_states_class_1}) and (\ref{eq:accepting_states_class_2}). 
Using these equations the probability, for either class type, that an individual 
is not lost in the system is given by:

\begin{equation*}
    P(L'_1) = \sum_{(u,v) \, \in S_A^{(1)}} \pi(u,v) \hspace{2cm}
    P(L'_2) = \sum_{(u,v) \, \in S_A^{(2)}} \pi(u,v)
\end{equation*}
 
Thus, by using these values as the coefficient of equation 
(\ref{eq:overall_waiting_time_coeff}) the resultant equation can be used to get 
the overall waiting time. 

\begin{equation}\label{eq:overall_waiting_time}
    W = \frac{\lambda_1 P(L'_1)}{\lambda_2 P(L'_2) + \lambda_1 P(L'_1)} W^{(1)} + 
    \frac{\lambda_2 P(L'_2)}{\lambda_2 P(L'_2) + \lambda_1 P(L'_1)} W^{(2)}
\end{equation}


\subsubsection{Blocking time}

% TODO: Possibly replace the contents of this section with the ones for the 
% closed form formula of the blocking time
% Currently the direct approach one is shown

Unlike the waiting time, the blocking time is only calculated for individuals of
the second type.  
That is because individuals of the first type cannot be blocked. 
Thus, one only needs to consider the pathway of type 2 individuals to get the 
mean blocking time of the system. 
The set of states where individuals can be blocked is defined as:

\begin{equation} \label{eq:set_of_blocking_states}
    S_b = \{(u,v) \in S \; | \; u > 0\}
\end{equation}
 
In order to not consider individuals that will be lost to the system, the set of 
accepting states needs to be taken into consideration. 
As defined in section \ref{sec:waiting-time}, the set of accepting states is 
given by \ref{eq:accepting-states-class-2}:

\begin{equation*}
    S_A^{(2)}=
    \begin{cases}
        \{(u, v) \in S \; | \; u < M \} & \textbf{if } T \leq N\\
        \{(u, v) \in S \; | \; v < N \} & \textbf{otherwise}
    \end{cases}
\end{equation*}

% For the waiting time formula in sections \ref{sec:recursive-waiting-time-others}
% and \ref{sec:recursive-waiting-time-ambulance}
% the mean sojourn time for each state was considered,
% ignoring any arrivals.
% Here, the same approach is used but ignoring only class 2
% arrivals. That is because for the waiting time formula, once an individual enters 
% the service centre (i.e. starts waiting) any individual arriving after them will 
% not affect their
% pathway. That is not the case for blocking time. When a class 2 individual is 
% blocked, 
% any class 1 individual that arrives will cause the blocked individual to remain 
% blocked for more time. Therefore, class 1 arrivals are considered here:

The mean sojourn time for each state is given by the inverse of the out-flow of
that state.
However, whenever a type 2 individual arrives at the system, no subsequent 
arrival of another type 2 individual can affect its pathway or total time in 
the system.
Therefore, looking at the mean time in the system from the perspective of an 
individual of the second type, all such type 2 arrivals need to be ignored.
Note here that this is not the case for individuals of the first type.
Whenever a type 2 individual is blocked and a type 1 individual arrives the type
2 individuals will remain blocked for some additional amount of time.
Thus, the mean time that a type 2 individual spends at each state is given by:

\begin{equation}\label{eq:time_in_state_blocking_time}
    c(u,v) = 
    \begin{cases}
        \frac{1}{\min(v,C) \mu}, & \text{if } v = N\\
        \frac{1}{\lambda_1 + \min(v,C) \mu}, & \text{otherwise}
    \end{cases}
\end{equation}
 
In equation \ref{eq:time_in_state_blocking_time}, both service completions and 
class 1 arrivals are considered. 
Thus, from a blocked individual's perspective whenever the system moves from one 
state \((u,v)\)
to another state it can either:

\begin{itemize}
    \item be because of a service being completed: we will denote the probability 
    of this happening by \(p_s(u,v)\). 
    \item be because of an arrival of an individual of class 1: denoting such 
    probability by \(p_o(u,v)\).
\end{itemize}
The probabilities are given by:

\begin{equation*}
    p_s(u,v) = \frac{\min(v,C)\mu}{\lambda_1 + \min(v,C)\mu}, \qquad
    p_o(u,v) = \frac{\lambda_1}{\lambda_1 + \min(v,C)\mu}
\end{equation*}


Having defined \(c(u,v)\) and \(S_b\) a formula for the blocking time that is
expected to occur at each state can be given by:

\begin{equation}\label{eq:general_blocking_time_at_each_state}
    b(u,v) = 
    \begin{cases} 
        0, & \textbf{if } (u,v) \notin S_b \\
        c(u,v) + b(u - 1, v), & \textbf{if } v = N = T\\
        c(u,v) + b(u, v-1), & \textbf{if } v = N \neq T \\
        c(u,v) + p_s(u,v) b(u-1, v) + p_o(u,v) b(u, v+1), & \textbf{if } u > 0 
        \textbf{ and } \vspace{-0.2cm} \\ 
        & \quad v = T \\
        c(u,v) + p_s(u,v) b(u, v-1) + p_o(u,v) b(u, v+1), & \textbf{otherwise} \\
    \end{cases}
\end{equation}

A direct approach will be used to solve this equation here. 
By enumerating all equations of (\ref{eq:general_blocking_time_at_each_state}) 
for all states \((u,v)\) that belong in \(S_b\) 
a system of linear equations arises where the unknown variables are all the 
\(b(u,v)\) terms. 
Note here that these equations correspond to all blocking states as defined in
\ref{eq:set_of_blocking_states}. 
Equations that correspond to non-blocking states have a value of \(0\) as 
defined in \ref{eq:general_blocking_time_at_each_state}
The general form of the equation in terms of \(C,T,N \text{ and } M\) is given by: 

\begin{align}
    b(1,T) \quad &= \quad c(1, T) + p_o b(1, T + 1) \label{eq:first_eq_of_blocking_general}\\
    b(1,T + 1) \quad &= \quad c(1, T + 1) + p_s b(1, T) + p_o b(1, T + 1) \\
    b(1,T + 2) \quad &= \quad c(1, T + 2) + p_s b(1, T + 1) + p_o b(1, T + 3) \\
    & \ \, \vdots \nonumber \\
    b(1, N) \quad &= \quad c(1, N) + b(1, N - 1) \\
    b(2, T) \quad &= \quad c(2, T) + p_s b(1, T) + p_o b(2, T + 1) \\
    b(2, T + 1) \quad &= \quad c(2, T + 1) + p_s b(2, T) + p_o b(2, T + 2) \\
    & \ \, \vdots \nonumber \\
    b(M - 1, N) \quad &= \quad c(M, N - 1) + b(M, N-1) \\ 
    b(M, T) \quad &= \quad c(T, N) + p_s b(T-1, N) + p_o b(T, N+1) \\
    & \ \, \vdots \nonumber \\
    b(M, N) \quad &= \quad c(M, N) + b(M, N-1) \label{eq:last_eq_of_blocking_general}
\end{align}

The equivalent matrix notation of the linear system of equations 
(\ref{eq:first_eq_of_blocking_general}) - (\ref{eq:last_eq_of_blocking_general})
is given by \(Zx=y\), where:
\begin{equation}\label{eq:general_algebaric_approach_blocking_time}
    \scalebox{0.73}{\(
        Z = 
        \begin{pmatrix}
            -1 & p_o & 0 & \dots & 0 & 0 & 0 & 0 & 0 & \dots & 0 & 0 \\ %(1,T)
            p_s & -1 & p_o & \dots & 0 & 0 & 0 & 0 & 0 & \dots & 0 & 0 \\ %(1,T+1)
            0 & p_s & -1 & \dots & 0 & 0 & 0 & 0 & 0 & \dots & 0 & 0 \\ %(1,T+2)
            \vdots & \vdots & \vdots & \ddots & \vdots & \vdots & \vdots & 
            \vdots & \vdots & \ddots & \vdots & \vdots \\ 
            0 & 0 & 0 & \dots & 1 & -1 & 0 & 0 & 0 & \dots & 0 & 0 \\ %(1,N)
            p_s & 0 & 0 & \dots & 0 & 0 & -1 & p_o & 0 & \dots & 0 & 0 \\ %(2,T)
            0 & 0 & 0 & \dots & 0 & 0 & p_s & -1 & p_o & \dots & 0 & 0 \\ %(2,T+1)
            \vdots & \vdots & \vdots & \ddots & \vdots & \vdots & \vdots & 
            \vdots & \vdots & \ddots & \vdots & \vdots \\ 
            0 & 0 & 0 & \dots & 0 & 0 & 0 & 0 & 0 & \dots & 1 & -1 \\ %(M,N)
        \end{pmatrix},
        x = 
        \begin{pmatrix}
            b(1,T) \\
            b(1,T+1) \\
            b(1,T+2) \\
            \vdots \\
            b(1,N) \\
            b(2,T) \\
            b(2,T+1) \\
            \vdots \\
            b(M,N) \\
        \end{pmatrix}, 
        y= 
        \begin{pmatrix}
            -c(1,T) \\
            -c(1,T+1) \\
            -c(1,T+2) \\
            \vdots \\
            -c(1,N) \\
            -c(2,T) \\
            -c(2,T+1) \\
            \vdots \\
            -c(M,N) \\
        \end{pmatrix}
    \)}
\end{equation}

Thus, having calculated the mean blocking time for all blocking states \(b(u,v)\), 
it only remains to put them together in a formula.
The resultant formula for the mean blocking time is given by:

\begin{equation}\label{eq:algebraic_blocking_time}
    B = \frac{\sum_{(u,v) \in S_A} \pi_{(u,v)} \; b(u,v)}{\sum_{(u,v) \in S_A} 
    \pi_{(u,v)}}
\end{equation}



To illustrate how the described formula works consider a Markov model where 
\(C=2, T=2, N=4, M=2\) (figure \ref{fig:example-algeb-blocking}). 
The equations that correspond to such a model are shown in 
(\ref{eq:first_eq_of_blocking_example})-(\ref{eq:last_eq_of_blocking_example}) 
and their equivalent matrix notation form is shown in 
\ref{eq:example_algebaric_approach_blocking_time}.

\begin{minipage}{.5\textwidth}
    \begin{figure}[H]
        \scalebox{0.6}{\begin{tikzpicture}[-, node distance = 1cm, auto]
\node[state] (u0v0) {(0,0)};
\node[state, right=of u0v0] (u0v1) {(0,1)};
\draw[->](u0v0) edge[bend left] node {\( \Lambda \)} (u0v1);
\draw[->](u0v1) edge[bend left] node {\(\mu \)} (u0v0);
\node[state, right=of u0v1] (u0v2) {(0,2)};
\draw[->](u0v1) edge[bend left] node {\( \Lambda \)} (u0v2);
\draw[->](u0v2) edge[bend left] node {\(2\mu \)} (u0v1);
\node[state, below=of u0v2] (u1v2) {(1,2)};
\draw[->](u0v2) edge[bend left] node {\( \lambda_2 \)} (u1v2);
\draw[->](u1v2) edge[bend left] node {\(2\mu \)} (u0v2);
\node[state, below=of u1v2] (u2v2) {(2,2)};
\draw[->](u1v2) edge[bend left] node {\( \lambda_2 \)} (u2v2);
\draw[->](u2v2) edge[bend left] node {\(2\mu \)} (u1v2);
\node[state, right=of u0v2] (u0v3) {(0,3)};
\draw[->](u0v2) edge[bend left] node {\( \lambda_1 \)} (u0v3);
\draw[->](u0v3) edge[bend left] node {\(2\mu \)} (u0v2);
\node[state, right=of u1v2] (u1v3) {(1,3)};
\draw[->](u1v2) edge[bend left] node {\( \lambda_1 \)} (u1v3);
\draw[->](u1v3) edge[bend left] node {\(2\mu \)} (u1v2);
\draw[->](u0v3) edge node {\( \lambda_2 \)} (u1v3);
\node[state, right=of u2v2] (u2v3) {(2,3)};
\draw[->](u2v2) edge[bend left] node {\( \lambda_1 \)} (u2v3);
\draw[->](u2v3) edge[bend left] node {\(2\mu \)} (u2v2);
\draw[->](u1v3) edge node {\( \lambda_2 \)} (u2v3);
\node[state, right=of u0v3] (u0v4) {(0,4)};
\draw[->](u0v3) edge[bend left] node {\( \lambda_1 \)} (u0v4);
\draw[->](u0v4) edge[bend left] node {\(2\mu \)} (u0v3);
\node[state, right=of u1v3] (u1v4) {(1,4)};
\draw[->](u1v3) edge[bend left] node {\( \lambda_1 \)} (u1v4);
\draw[->](u1v4) edge[bend left] node {\(2\mu \)} (u1v3);
\draw[->](u0v4) edge node {\( \lambda_2 \)} (u1v4);
\node[state, right=of u2v3] (u2v4) {(2,4)};
\draw[->](u2v3) edge[bend left] node {\( \lambda_1 \)} (u2v4);
\draw[->](u2v4) edge[bend left] node {\(2\mu \)} (u2v3);
\draw[->](u1v4) edge node {\( \lambda_2 \)} (u2v4);
\end{tikzpicture}}
        \caption{
            \centering{Example of Markov chain with \(C=2, T=2, N=4, M=2\)}
        }
        \label{fig:example-algeb-blocking}
    \end{figure}
\end{minipage}
\begin{minipage}{.43\textwidth}
    \begin{align}
        b(1,2) &= c(1,2) + p_o b(1,3) \label{eq:first_eq_of_blocking_example} \\
        b(1,3) &= c(1,3) + p_s b(1,2) \nonumber \\ &+ p_o b(1,4) \\
        b(1,4) &= c(1,4) + b(1,3) \\
        b(2,2) &= c(2,2) + p_s b(1,2) \nonumber \\ &+ p_o b(2,3) \\
        b(2,3) &= c(2,3) + p_s b(2,2) \nonumber \\ &+ p_o b(1,4) \\
        b(2,4) &= c(2,4) + b(2,3) \label{eq:last_eq_of_blocking_example}
    \end{align}
\end{minipage}

\begin{equation}\label{eq:example_algebaric_approach_blocking_time}
    Z=
    \begin{pmatrix}
        -1 & p_o & 0 & 0 & 0 & 0 \\ %(1,2)
        p_s & -1 & p_o & 0 & 0 & 0 \\ %(1,3)
        0 & 1 & -1 & 0 & 0 & 0 \\ %(1,4)
        p_s & 0 & 0 & -1 & p_o & 0\\ %(2,2)
        0 & 0 & 0 & p_s & -1 & p_o \\ %(2,3)
        0 & 0 & 0 & 0 & 1 & -1 \\ %(2,4)
    \end{pmatrix},
    x=
    \begin{pmatrix}
        b(1,2) \\
        b(1,3) \\
        b(1,4) \\
        b(2,2) \\
        b(2,3) \\
        b(2,4) \\
    \end{pmatrix}, 
    y=
    \begin{pmatrix}
        -c(1,2) \\
        -c(1,3) \\
        -c(1,4) \\
        -c(2,2) \\
        -c(2,3) \\
        -c(2,4) \\
    \end{pmatrix}
\end{equation}


\subsubsection{Proportion of individuals within target}

Another performance measure that needs to be taken into consideration is the 
proportion of individuals whose waiting and service times lie within a specified 
time target.
In order to consider such measure though one would need to obtain the 
distribution of time in the system for all individuals. 
The complexity of such task lies on the fact that different individuals arrive 
at different states of the Markov model. 
Consider the case when an arrival occurs when the model is at a specific state.

\paragraph{Time distribution at specific state (1 server):}

\begin{figure}[ht]
    \centering
    \scalebox{0.75}{
        \begin{tikzpicture}[-, node distance = 1cm, auto]
\node[state] (u0v0) {(0,0)};
\node[state, right=of u0v0] (u0v1) {(0,1)};
\draw[->](u0v0) edge[bend left] node {\( \Lambda \)} (u0v1);
\draw[->](u0v1) edge[bend left] node {\(\mu \)} (u0v0);
\node[state, right=of u0v1] (u0v2) {(0,2)};
\draw[->](u0v1) edge[bend left] node {\( \Lambda \)} (u0v2);
\draw[->](u0v2) edge[bend left] node {\(\mu \)} (u0v1);
\node[state, below=of u0v2] (u1v2) {(1,2)};
\draw[->](u0v2) edge[bend left] node {\( \lambda_2 \)} (u1v2);
\draw[->](u1v2) edge[bend left] node {\(\mu \)} (u0v2);
\node[state, below=of u1v2] (u2v2) {(2,2)};
\draw[->](u1v2) edge[bend left] node {\( \lambda_2 \)} (u2v2);
\draw[->](u2v2) edge[bend left] node {\(\mu \)} (u1v2);
\node[state, right=of u0v2] (u0v3) {(0,3)};
\draw[->](u0v2) edge[bend left] node {\( \lambda_1 \)} (u0v3);
\draw[->](u0v3) edge[bend left] node {\(\mu \)} (u0v2);
\node[state, right=of u1v2] (u1v3) {(1,3)};
\draw[->](u1v2) edge[bend left] node {\( \lambda_1 \)} (u1v3);
\draw[->](u1v3) edge[bend left] node {\(\mu \)} (u1v2);
\draw[->](u0v3) edge node {\( \lambda_2 \)} (u1v3);
\node[state, right=of u2v2] (u2v3) {(2,3)};
\draw[->](u2v2) edge[bend left] node {\( \lambda_1 \)} (u2v3);
\draw[->](u2v3) edge[bend left] node {\(\mu \)} (u2v2);
\draw[->](u1v3) edge node {\( \lambda_2 \)} (u2v3);
\node[state, right=of u0v3] (u0v4) {(0,4)};
\draw[->](u0v3) edge[bend left] node {\( \lambda_1 \)} (u0v4);
\draw[->](u0v4) edge[bend left] node {\(\mu \)} (u0v3);
\node[state, right=of u1v3] (u1v4) {(1,4)};
\draw[->](u1v3) edge[bend left] node {\( \lambda_1 \)} (u1v4);
\draw[->](u1v4) edge[bend left] node {\(\mu \)} (u1v3);
\draw[->](u0v4) edge node {\( \lambda_2 \)} (u1v4);
\node[state, right=of u2v3] (u2v4) {(2,4)};
\draw[->](u2v3) edge[bend left] node {\( \lambda_1 \)} (u2v4);
\draw[->](u2v4) edge[bend left] node {\(\mu \)} (u2v3);
\draw[->](u1v4) edge node {\( \lambda_2 \)} (u2v4);
\end{tikzpicture}
    }
    \caption{Example Markov model \(C=1, T=2, N=4, M=2\)}
    \label{fig:distribution_of_time_at_specific_state_1_server}
\end{figure}

Consider the Markov model of figure 
\ref{fig:distribution_of_time_at_specific_state_1_server} with one server and a 
threshold of two individuals. 
Assume that an individual of the first type arrives when the model is at state 
\((0,3)\), thus forcing the model to move to state \((0,4)\). 
The distribution of the time needed for the specified individual to exit the 
system from state \((0,4)\) is given by the sum of exponentially distributed 
random variables with the same parameter \(\mu\). 
The sum of such random variables forms an Erlang distribution which is defined 
by the number of random variables that are added and their exponential 
parameter.
Note here that these random variables represent the individual's pathway from 
the perspective of the individual. 
Thus, \(X_i\) represents the time that it takes to move from the 
\(i^{\text{th}}\) position of the queue to the \((i-1)^{\text{th}}\) position 
(i.e. for someone in front of them to finish their service) and \(X_0\) is the 
time it takes to move from having a service to exiting the system.

\begin{align}
    (0,4) \Rightarrow \quad & X_3 \sim Exp(\mu) \nonumber \\
    (0,3) \Rightarrow \quad & X_2 \sim Exp(\mu) \nonumber \\
    (0,2) \Rightarrow \quad & X_1 \sim Exp(\mu) \nonumber \\
    (0,1) \Rightarrow \quad & X_0 \sim Exp(\mu) \nonumber \\
    S = X_3 + X_2 + & X_1 + X_0 = Erlang(4, \mu)
\end{align}

Thus, the waiting and service time of an individual in the model of figure 
\ref{fig:distribution_of_time_at_specific_state_1_server} can be captured by an 
erlang distributed random variable. 
The general CDF of the erlang distribution \(Erlang(k, \mu)\) is given by:

\begin{equation} \label{eq:cdf_erlang}
    P(S < t) = 1 - \sum_{i=0}^{k-1} \frac{1}{i!} e^{-\mu t} (\mu t)^i
\end{equation}

Unfortunately, the erlang distribution can only be used for the sum of 
identically distributed random variables from the exponential distribution. 
Therefore, this approach cannot be used when one of the random variables has a 
different parameter than the others. 
In fact the only case where it can be used is only when the number of servers 
are \(C=1\), or when an individual arrives and goes straight to service 
(i.e. when there is no other individual waiting and there is an empty server).


\paragraph{Time distribution at a state (multiple servers):}

\begin{figure}[h]
    \centering
    \scalebox{0.75}{\begin{tikzpicture}[-, node distance = 1cm, auto]
\node[state] (u0v0) {(0,0)};
\node[state, right=of u0v0] (u0v1) {(0,1)};
\draw[->](u0v0) edge[bend left] node {\( \Lambda \)} (u0v1);
\draw[->](u0v1) edge[bend left] node {\(\mu \)} (u0v0);
\node[state, right=of u0v1] (u0v2) {(0,2)};
\draw[->](u0v1) edge[bend left] node {\( \Lambda \)} (u0v2);
\draw[->](u0v2) edge[bend left] node {\(2\mu \)} (u0v1);
\node[state, below=of u0v2] (u1v2) {(1,2)};
\draw[->](u0v2) edge[bend left] node {\( \lambda_2 \)} (u1v2);
\draw[->](u1v2) edge[bend left] node {\(2\mu \)} (u0v2);
\node[state, below=of u1v2] (u2v2) {(2,2)};
\draw[->](u1v2) edge[bend left] node {\( \lambda_2 \)} (u2v2);
\draw[->](u2v2) edge[bend left] node {\(2\mu \)} (u1v2);
\node[state, right=of u0v2] (u0v3) {(0,3)};
\draw[->](u0v2) edge[bend left] node {\( \lambda_1 \)} (u0v3);
\draw[->](u0v3) edge[bend left] node {\(2\mu \)} (u0v2);
\node[state, right=of u1v2] (u1v3) {(1,3)};
\draw[->](u1v2) edge[bend left] node {\( \lambda_1 \)} (u1v3);
\draw[->](u1v3) edge[bend left] node {\(2\mu \)} (u1v2);
\draw[->](u0v3) edge node {\( \lambda_2 \)} (u1v3);
\node[state, right=of u2v2] (u2v3) {(2,3)};
\draw[->](u2v2) edge[bend left] node {\( \lambda_1 \)} (u2v3);
\draw[->](u2v3) edge[bend left] node {\(2\mu \)} (u2v2);
\draw[->](u1v3) edge node {\( \lambda_2 \)} (u2v3);
\node[state, right=of u0v3] (u0v4) {(0,4)};
\draw[->](u0v3) edge[bend left] node {\( \lambda_1 \)} (u0v4);
\draw[->](u0v4) edge[bend left] node {\(2\mu \)} (u0v3);
\node[state, right=of u1v3] (u1v4) {(1,4)};
\draw[->](u1v3) edge[bend left] node {\( \lambda_1 \)} (u1v4);
\draw[->](u1v4) edge[bend left] node {\(2\mu \)} (u1v3);
\draw[->](u0v4) edge node {\( \lambda_2 \)} (u1v4);
\node[state, right=of u2v3] (u2v4) {(2,4)};
\draw[->](u2v3) edge[bend left] node {\( \lambda_1 \)} (u2v4);
\draw[->](u2v4) edge[bend left] node {\(2\mu \)} (u2v3);
\draw[->](u1v4) edge node {\( \lambda_2 \)} (u2v4);
\end{tikzpicture}}
    \caption{Example Markov model \(C=2, T=2, N=4, M=2\)}
    \label{fig:distribution_of_time_at_specific_state_2_servers}
\end{figure}

Figure \ref{fig:distribution_of_time_at_specific_state_2_servers} represents the 
same Markov model as figure 
\ref{fig:distribution_of_time_at_specific_state_1_server} with the only 
exception that there are 2 servers here. 
By applying the same logic, assuming that an individual arrives at state 
\((0,4)\), the sum of the following random variables arises.

\begin{align}
    (0,4) \Rightarrow \quad & X_2 \sim Exp(2\mu) \nonumber \\
    (0,3) \Rightarrow \quad & X_1 \sim Exp(2\mu) \\
    (0,2) \Rightarrow \quad & X_0 \sim Exp(\mu) \nonumber
\end{align}

Since these exponentially distributed random variables do not share the same 
parameter, an erlang distribution cannot be used. 
In fact, the problem can now be viewed either as the sum of exponentially 
distributed random variables with different parameters or as the sum of 
erlang distributed random variables.
The sum of erlang distributed random variables is said to follow the 
hypoexponential distribution. 
The hypoexponential distribution is defined with two vectors of size equal
to the number of Erlang random variables \cite{Akkouchi2008}, \cite{Smaili2013}. 
The vector \(\vec{r}\) contains all the \(k\)-values of the erlang distributions 
and \(\vec{\lambda}\) is a vector of the distinct parameters as illustrated in
equation \ref{eq:connection_between_Hypoexponential_Erlang}.

\begin{equation}\label{eq:connection_between_Hypoexponential_Erlang}
    \begin{rcases}
        Erlang(k_1, \lambda_1) \\
        Erlang(k_2, \lambda_2) \\
        \hspace{1cm} \vdots \\
        Erlang(k_n, \lambda_n)
    \end{rcases}
    Hypo(
        \underbrace{(k_1, k_2, \dots k_n)}_{\vec{k}}, 
        \underbrace{(\lambda_1, \lambda_2, \dots \lambda_n)}_{\vec{\lambda}}
    )
\end{equation}

Equivalently, for this particular example:
\begin{align} \label{eq:multiple_servers_distribution_example}
    \begin{rcases}
        \begin{rcases}
            \scriptstyle{X_2 \sim Exp(2\mu)} \\
            \scriptstyle{X_1 \sim Exp(2\mu)}
        \end{rcases}
        \scriptstyle{X_1 + X_2 = S_1 \sim Erlang(2, 2\mu)} \\
        \scriptstyle{X_0 \sim Exp(\mu) \Rightarrow 
        \hspace{1cm} X_0 = S_2 \sim Erlang(1, \mu)}
    \end{rcases}
    \scriptstyle{S_1 + S_2 = H \sim Hypo((2,1), (2\mu, \mu))}
\end{align}

Therefore, the CDF of this distribution can be used to get the probability of 
the time in spent in the system being less than a given target.
The general CDF of the hypoexponential distribution \(Hypo(\vec{r}, 
\vec{\lambda})\), is given by the following expression \cite{Favaro2010}:

\begin{align} \label{eq:general_cdf_hypoexponential}
    & P(H < t) = 1 - \left( \prod_{j=1}^{\mid \vec{r} \mid} \lambda_j^{r_j} \right) 
    \sum_{k=1}^{\mid \vec{r} \mid} \sum_{l=1}^{r_k} \frac{\Psi_{k,l}(-\lambda_k)t^{r_k - l} 
    e^{-\lambda_k t}}
    {(r_k - l)! (l - 1)!} \nonumber \\ 
    & \textbf{where} \qquad \Psi_{k,l}(t) = - \frac{\partial^{l - 1}}
    {\partial t ^{l - 1}} \left( \prod_{j = 0, j \neq k}^{\mid \vec{r} \mid} 
    (\lambda_j + t)^{-r_j} \right) \nonumber \\
    & \textbf{and} \quad \qquad \lambda_0 = 0, r_0 = 1
\end{align}


The computation of the derivative makes equation 
\ref{eq:general_cdf_hypoexponential} computationally expensive. 
In \cite{Legros2015} an alternative linear version of that CDF is explored via 
matrix analysis, and is given by the following formula:

\begin{equation} \label{eq:linear_general_cdf_hypoexponential}
    \begin{split}
        F(x) = &1 - \sum_{k=1}^{n} \sum_{l=0}^{k-1} (-1)^{k-1} \binom{n}{k} 
            \binom{k-1}{l} \sum_{j=1}^{n} \sum_{s=1}^{j-1} e^{-x \lambda_s} 
            \prod_{l=1}^{s-1} \left( \frac{\lambda_l}{\lambda_l - \lambda_s} \right)
            ^ {k_s} \\
        & \times \sum_{s < a_1 < \dots < a_{l-1} < j} 
            \left( \frac{\lambda_s}{\lambda_s - \lambda_{a_1}} \right) ^ {k_s}
            \prod_{m=s+1}^{a_1-1} \left( \frac{\lambda_m}{\lambda_m - 
            \lambda_{a_1}}\right) ^ {k_m} \\  
        & \times \prod_{n=a_1}^{a_2-1} \left( \frac{\lambda_n}{\lambda_n - 
            \lambda_{a_2}}\right) ^ {k_n} \dots 
            \prod_{r=a_l-1}^{j-1} \left( \frac{\lambda_r}{\lambda_r - 
            \lambda_{a_j}}\right) ^ {k_r}  
            \sum_{q=0}^{k_s - 1} \frac{((\lambda_s - \lambda_{a_1})x)^q}{q!}, \\
        & \text{for } x \geq 0
    \end{split}
\end{equation}


\paragraph{Specific CDF of hypoexponential distribution}
Equations \ref{eq:general_cdf_hypoexponential} and 
\ref{eq:linear_general_cdf_hypoexponential} refers to the general CDF of the
hypoexponential distribution where the size of the vector parameters can be of
any size \cite{Favaro2010}.
In the Markov chain models described in figures 
\ref{fig:distribution_of_time_at_specific_state_1_server} and 
\ref{fig:distribution_of_time_at_specific_state_2_servers} the parameter vectors 
of the hypoexponential distribution are of size two, and in fact, for any 
possible version of the investigated Markov chain model the vectors can only be 
of size two.
This is true since for any dimensions of this Markov chain model there will 
always be at most two distinct exponential parameters; the parameter for 
finishing a service (\(\mu\)) and the parameter for moving forward in the queue 
(\(C \mu\)). 
For the special case of \(C=1\) the hypoexponential distribution will not be 
used as this is equivalent to an erlang distribution.
Therefore, by fixing the sizes of \(\vec{r}\) and \(\vec{\lambda}\) to 2, the 
following specific expression for the CDF of the hypoexponential distribution
arises, where the derivative is removed:


\begin{align} \label{eq:specific_cdf_hypoexponential}
    & P(H < t) = 1 - \left( \prod_{j=1}^{\mid \vec{r} \mid} \lambda_j^{r_j} \right) 
    \sum_{k=1}^{\mid \vec{r} \mid} \sum_{l=1}^{r_k} \frac{\Psi_{k,l}(-\lambda_k)t^{r_k - l} 
    e^{-\lambda_k t}}{(r_k - l)! (l - 1)!} \nonumber \\ 
    & \textbf{where} \qquad \Psi_{k,l}(t) = 
    \begin{cases} 
        \frac{(-1)^{l} (l-1)!}{\lambda_2} \left[\frac{1}{t^l} - \frac{1}
        {(t + \lambda_2)^l}\right] , & k=1 \\
        - \frac{1}{t (t + \lambda_1)^{r_1}}, & k=2
    \end{cases} \nonumber \\
    & \textbf{and} \quad \qquad \lambda_0 = 0, r_0 = 1
\end{align}

Note here that the only difference between equations
\ref{eq:general_cdf_hypoexponential} and \ref{eq:specific_cdf_hypoexponential} 
is the \(\Psi\) function. 
The next part proves that the expression for \(\Psi\) can be simplified for the 
cases of \(k = 1,2\). 
Equation \ref{eq:hypoexponential_expression_to_proof} shows the expression to 
be proved.

\begin{equation} \label{eq:hypoexponential_expression_to_proof}
    \Psi_{(k,l)}(t) = - \frac{\partial^{l - 1}}{\partial t ^{l - 1}} 
    \left( \prod_{j = 0, j \neq k}^{\mid \vec{r} \mid} (\lambda_j + t)^{-r_j} \right) = 
    \begin{cases} 
        \frac{(-1)^{l} (l-1)!}{\lambda_2} \left[\frac{1}{t^l} - \frac{1}
        {(t + \lambda_2)^l}\right] , & k=1 \\
        - \frac{1}{t (t + \lambda_1)^{r_1}}, & k=2
    \end{cases}
\end{equation}



\paragraph{Proof of equation \ref{eq:hypoexponential_expression_to_proof}}
 
This section aims to show that there exists a simplified version of equation 
\ref{eq:general_cdf_hypoexponential} that is specific to the proposed Markov 
model.
Function \(\Psi\) is defined using the parameter \(t\) and the variables \(k\) 
and \(l\).
Given the Markov model, the range of values that \(k\) and \(l\) can take can be
bounded.
First of all, from the range of the double summation in equation 
\ref{eq:general_cdf_hypoexponential}, it can be seen that 
\(k = 1, 2, \dots, \mid \vec{r} \mid\).
Now, \(\mid \vec{r} \mid\) represents the size of the parameter vectors that, 
for the Markov model, will always be 2. 
That is because, for all the exponentially distributed random variables that are
added together to form the new distribution, there only two distinct parameters,
thus forming two erlang distributions. Therefore:

\begin{equation*}
    k = 1, 2
\end{equation*}

By observing equation \ref{eq:general_cdf_hypoexponential} once more, the range
of values that \(l\) takes are \(l = 1, 2, \dots, r_k\), where \(r_1\) is 
subject to the individual's position in the queue and \(r_2 = 1\).
In essence, the hypoexponential distribution will be used with these bounds:

\begin{align}
    k = 1 & \qquad \Rightarrow \qquad l = 1, 2, \dots, r_1 \nonumber \\
    k = 2 & \qquad \Rightarrow \qquad l = 1
\end{align}

Thus the left hand side of equation \ref{eq:hypoexponential_expression_to_proof} 
needs only to be defined for these bounds. 
The specific hypoexponential distribution investigated here is of the form
\(Hypo((r_1, 1)(\lambda_1, \lambda_2))\).
Note the initial conditions \(\lambda_0=0, r_0=1\) defined in equation 
\ref{eq:general_cdf_hypoexponential} also hold here.
Thus the proof is split into two parts, for \(k=1\) and \(k=2\).



\begin{itemize}
    \item \(k = 2, l = 1\)
    \begin{equation*}
        \begin{split}
            LHS &= - \frac{\partial^{1-1}}{\partial t^{1-1}} 
            \left( \prod_{j=0, j \neq 2}^{2} (\lambda_j + t)^{-r_j} \right) \\
            &=-\left( (\lambda_0 + t)^{-r_0} \times (\lambda_1 + t)^{-r_1} \right) \\
            &=-\left( t^{-1} \times (\lambda_1 + t)^{-r_1} \right) \\ 
            &= - \frac{1}{t(t + \lambda_1)^{r_1}} \\
            & \hspace{7cm} \square
        \end{split}
    \end{equation*}
    \item \(k = 1, l = 1, \dots, r_1\)
    \begin{equation*}
        \begin{split}
            LHS &= -\frac{\partial^{l-1}}{\partial t^{l-1}} 
            \left( \prod_{j=0, j \neq 1}^{2} (\lambda_j + t)^{-r_j} \right) \\
            &= -\frac{\partial^{l-1}}{\partial t^{l-1}}
            \left( (\lambda_o + t)^{-r_0} \times (\lambda_2 + t)^{-r_2} \right) \\
            &= -\frac{\partial^{l-1}}{\partial t^{l-1}}
            \left( \frac{1}{t(t + \lambda_2)}\right)
        \end{split}
    \end{equation*}
    In essence, it only remains to show that:
    \[-\frac{\partial^{l-1}}{\partial t^{l-1}} 
    \left( \frac{1}{t(t + \lambda_2)}\right) = \frac{(-1)^{l} (l-1)!}{\lambda_2}
    \left[\frac{1}{t^l} - \frac{1}{(t + \lambda_2)^l}\right]\]
    
    \textbf{Proof by Induction:}
    \begin{enumerate}
        \item Base case (\(l=1\)):
        \begin{equation*}
            \begin{split}
                LHS &= -\frac{\partial^{1-1}}{\partial t^{1-1}} 
                \left( \frac{1}{t(t + \lambda_2)}\right) = 
                - \frac{1}{t(t + \lambda_2)} \\
                RHS &= \frac{(-1)^{1} (1-1)!}{\lambda_2}
                \left[\frac{1}{t^1} - \frac{1}{(t + \lambda_2)^1}\right] \\
                &= - \frac{t + \lambda_2 - t}{\lambda_2 t (t + \lambda_2)} \\
                &= - \frac{1}{t (t + \lambda_2)} \\
                LHS &= RHS
            \end{split}
        \end{equation*}
        \item Assume true for \(l = x\):
        \begin{equation*}
            -\frac{\partial^{x-1}}{\partial t^{x-1}} 
            \left( \frac{1}{t(t + \lambda_2)}\right) = 
            \frac{(-1)^{x} (x-1)!}{\lambda_2}
            \left[\frac{1}{t^x} - \frac{1}{(t + \lambda_2)^x}\right]
        \end{equation*}
        \item Prove true for \(l = x + 1\). Need to show that:
        \[ 
            \frac{\partial^x}{\partial t ^ x} 
            \left( \frac{-1}{t (t + \lambda_2)} \right) = 
            \frac{(-1)^{x + 1} (x)!}{\lambda_2}
            \left[ \frac{1}{t^{x+1}}-\frac{1}{(t + \lambda_2)^{x+1}} \right] 
        \]
        \begin{equation*}
            \begin{split}
                LHS &= \frac{\partial}{\partial t}
                \left[ \frac{\partial^{x-1}}{\partial t ^ {x-1}} 
                \left( \frac{-1}{t (t + \lambda_2)} \right) \right] \\
                &= \frac{\partial}{\partial t} \left[
                    \frac{(-1)^x (x-1)!}{\lambda_2} \left(
                        \frac{1}{t^x} - \frac{1}{(t + \lambda_2)^x}
                    \right)
                \right] \\
                &= \frac{(-1)^x (x-1)!}{\lambda_2} \left(
                    \frac{(-x)}{t^{x+1}} - \frac{(-x)}{(t + \lambda_2)^x}
                \right) \\
                &= \frac{(-1)^x (x-1)! (-x)}{\lambda_2} \left(
                    \frac{1}{t^{x+1}} - \frac{1}{(t + \lambda_2)^x}
                \right) \\
                &= \frac{(-1)^{x+1} (x)!}{\lambda_2} \left(
                    \frac{1}{t^{x+1}} - \frac{1}{(t + \lambda_2)^x}
                \right) \\
                & = RHS \\
                & \hspace{7cm} \square
            \end{split}
        \end{equation*}
    \end{enumerate}
\end{itemize}

\paragraph{Proportion within target for both types of individuals}

Given the two CDFs of the Erlang and Hypoexponential distributions a new 
function has to be defined to decide which one to use among the two.
Based on the state of the model, there can be three scenarios when an individual
arrives.
\begin{enumerate}
    \item There is a free server and the individual does not have to wait
    \begin{equation*}
        X_{(u,v)} \sim Erlang(1, \mu) 
    \end{equation*}
    \item The individual arrives at a queue at the \(n^{th}\) position and the 
    model has \(C > 1\) servers
    \begin{equation*}
        X_{(u,v)} \sim Hypo((n, 1), (C \mu, \mu)) 
    \end{equation*}
    \item The individual arrives at a queue at the \(n^{th}\) position and the 
    model has \(C = 1\) servers
    \begin{equation*}
        X_{(u,v)} \sim Erlang(n + 1, \mu) 
    \end{equation*}
\end{enumerate}

Note here that for the first case \(Erlang(1, \mu)\) is equivalent to 
\(Exp(\mu)\). 
Consider \(X_{(u,v)}^{(1)}\) to be the distribution of type 1 individuals and
\(X_{(u,v)}^{(2)}\) the distribution of type 2 individuals, when arriving at 
state \((u,v)\) of the model.

\begin{equation}
    X_{(u,v)}^{(1)} \sim 
    \begin{cases}
        \textbf{Erlang}(v, \mu), & \textbf{if } C = 1 \textbf{ and } v>1 \\
        \textbf{Hypo} \left(
            \left[v - C, 1\right], \left[C \mu, \mu \right]
        \right), & \textbf{if } C > 1 \textbf{ and } v>C \\
        \textbf{Erlang}(1, \mu), & \textbf{if } v \leq C
    \end{cases}
\end{equation}

\begin{equation}
    X_{(u,v)}^{(2)} \sim 
    \begin{cases}
        \textbf{Erlang}(\min(v, T), \mu), & \textbf{if } C = 1
            \textbf{ and } v, T > 1 \\
        \textbf{Hypo}\left(
            \left[ \min(v, T) - C, 1 \right], \left[ C \mu, \mu \right]
        \right), & \textbf{if } C > 1 \textbf{ and } v, T  > C \\
        \textbf{Erlang}(1, \mu), & \textbf{if } v \leq C \textbf{ or } T \leq C
    \end{cases}
\end{equation}


Thus, the CDF of the random variables \(X_{(u,v)}^{(1)}\) and 
\(X_{(u,v)}^{(2)}\) can be calculated using equations \ref{eq:cdf_erlang} and 
\ref{eq:specific_cdf_hypoexponential}:

\begin{equation}
    P(X_{(u,v)}^{(1)} < t) = 
    \begin{cases}
        1 - \sum_{i=0}^{v-1} \frac{1}{i!} e^{-\mu t} (\mu t)^i, 
            & \textbf{if } C = 1 \\
            & \textbf{and } v>1 \\
        & \\
        1 - (\mu C)^{v-C} \mu  
            \sum_{k=1}^{\mid \vec{r} \mid} \sum_{l=1}^{r_k}
            \frac{\Psi_{k,l}(-\lambda_k)t^{r_k - l} 
            e^{-\lambda_k t}}{(r_k - l)! (l - 1)!},
            & \textbf{if } C > 1 \\
        \textbf{where } \vec{r}=(v - C, 1) \textbf{ and } 
            \vec{\lambda}=(C \mu, \mu) & \textbf{and } v > C \\
        & \\
        1 - e^{-\mu t},  & \textbf{if } v \leq C
    \end{cases}
\end{equation}


\begin{equation}
    P(X_{(u,v)}^{(2)} < t) = 
    \begin{cases}
        1 - \sum_{i=0}^{\min(v,T)-1} \frac{1}{i!} e^{-\mu t} (\mu t)^i,  
            & \textbf{if } C = 1 \\ 
            & \textbf{and } v, T > 1 \\
            & \\
        1 - (\mu C) ^ {\min(v,T) - C} \mu  & \textbf{if } C > 1 \\
        \qquad \times \sum_{k=1}^{\mid \vec{r} \mid} \sum_{l=1}^{r_k} 
            \frac{\Psi_{k,l}(-\lambda_k)t^{r_k - l} 
            e^{-\lambda_k t}}{(r_k - l)! (l - 1)!}, 
            & \textbf{and } v, T  > C \\
        \textbf{where } \vec{r}=(\min(v, T) - C, 1) \\
        \hspace{1.15cm} \vec{\lambda}=(C \mu, \mu) \\
        & \\
        1 - e^{-\mu t}, & \textbf{if } v \leq C \\ 
        & \textbf{or } T \leq C \\
    \end{cases}
\end{equation}


In addition, the set of accepting states for type 1 \(S_A^{(1)}\) and type 2 
\(S_A^{(2)}\) individuals defined in \ref{eq:accepting_states_class_1} and 
\ref{eq:accepting_states_class_2} are also needed here.
Note here that, \(S\) denotes the set of all states of the Markov chain model. 

\begin{align*}
    S_A^{(1)} &= \{(u, v) \in S \; | \; v < N \} \\
    S_A^{(2)} &=
    \begin{cases}
        \{(u, v) \in S \; | \; u < M \}, & \textbf{if } T \leq N \\
        \{(u, v) \in S \; | \; v < N \}, & \textbf{otherwise}
    \end{cases}
\end{align*}

The following formula uses the state probability vector \(\pi\) to get the 
weighted average of the probability below target of all states in the Markov
model.

\begin{equation}
    P(X^{(1)} < t) = \frac{\sum_{(u,v) \in S_A^{(1)}} P(X_{u,v}^{(1)} < t) 
    \pi_{u,v} }{\sum_{(u,v) \in S_A^{(1)}} \pi_{u,v}}
\end{equation}

\begin{equation}
    P(X^{(2)} < t) = \frac{\sum_{(u,v) \in S_A^{(2)}} P(X_{u,v}^{(2)} < t) 
    \pi_{u,v} }{\sum_{(u,v) \in S_A^{(2)}} \pi_{u,v}}
\end{equation}


\paragraph{Overall proportion within target}

The overall proportion of individuals for both types of individuals is given by 
the equivalent formula of equations (\ref{eq:overall_waiting_time_coeff}) and 
(\ref{eq:overall_waiting_time}).
The following formula uses the probability of lost individuals from both types
to get the weighted sum of the two probabilities.

\begin{equation*}
    P(L'_1) = \sum_{(u,v) \, \in S_A^{(1)}} \pi(u,v), \hspace{1.5cm}
    P(L'_2) = \sum_{(u,v) \, \in S_A^{(2)}} \pi(u,v)
\end{equation*}

\begin{align}\label{eq:overall_proportion_within_target}
    P(X < t) &= \frac{\lambda_1 P(L'_1)}{\lambda_2 P(L'_2)+\lambda_1 P(L'_1)} 
    P(X^{(1)} < t) \\
    &+ \frac{\lambda_2 P(L'_2)}{\lambda_2 P(L'_2) + \lambda_1 
    P(L'_1)} P(X^{(2)} < t)
\end{align}


\subsection{Example}
