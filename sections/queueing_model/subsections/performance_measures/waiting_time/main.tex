\subsubsection{Waiting time} \label{sec:waiting_time}

Waiting time is the amount of time that individuals from either class wait in 
waiting zone 1 so that they can receive their service. 
For a given set of parameters there are three different performance measures 
around the mean waiting time that can be calculated; the mean waiting time of
class 1 individuals, the mean waiting time of class 2 individuals and the 
overall mean waiting time. 

Since some of the individuals can be lost to the model, a new set of states 
needs to be defined; the set of \textit{accepting states}. 
That is the set of states that the model is able to accept a certain type of
individual. 
The set of accepting states for class 1 individuals is defined as:

\begin{equation}\label{eq:accepting_states_class_1}
    S_A^{(1)} = \{(u, v) \in S \; | \; v < N \}
\end{equation}

In essence, for class 1 individuals, this is the set of states that are not on 
the last column of states in the Markov chain.
Equivalently, the set of accepting states for class 2 individuals is defined as:

\begin{equation}\label{eq:accepting_states_class_2}
    S_A^{(2)}=
    \begin{cases}
        \{(u, v) \in S \; | \; u < M \}, & \textbf{if } T \leq N\\
        \{(u, v) \in S \; | \; v < N \}, & \textbf{otherwise}
    \end{cases}
\end{equation}

Note here that if the threshold is less than or equal the total capacity of the
system the set includes all states that are not on the last column of the 
Markov chain.
Otherwise, the set of accepting state is identical to 
(\ref{eq:accepting_states_class_1}). Thus, the expressions for the waiting times 
for class 1 and class 2 individuals are given by:

\begin{equation} \label{eq:closed_form_waiting_class_1}
    W^{(1)} = \frac{\sum_{\substack{(u,v) \, \in S_A^{(1)} \\ v \geq C}} 
    \frac{1}{C \mu} \times (v-C+1) \times \pi(u,v)}{\sum_{(u,v) \, 
    \in S_A^{(1)}} \pi(u,v)}
\end{equation}
    
\begin{equation}\label{eq:closed_form_waiting_class_2}
    W^{(2)} = \frac{\sum_{\substack{(u,v) \, \in S_A^{(2)} \\ min(v,T) \geq C}} 
    \frac{1}{C \mu} \times (\min(v+1,T)-C) \times \pi(u,v)}{\sum_{(u,v) \, 
    \in S_A^{(2)}} \pi(u,v)}
\end{equation}

Consequently, the overall waiting time can be estimated by a linear combination 
of \(W_1\) and \(W_2\). 
Thus, the overall waiting time can calculated by the following equation where 
\(c_1\) and \(c_2\) are the coefficients of the terms:

\begin{equation}\label{eq:overall_waiting_time_coeff}
    W = c_1 W^{(1)} + c_2 W^{(2)}
\end{equation}

The two coefficients represent the proportion of individuals of each type that 
did not get lost and traversed through the model.
Thus, one should account for the probability that an individual is lost to the 
system. 
This probability can be easily calculated by using the two sets of accepting 
states \(S_A^{(2)}\) and \(S_A^{(1)}\) defined in equations 
(\ref{eq:accepting_states_class_1}) and (\ref{eq:accepting_states_class_2}). 
Using these equations the probability, for either class type, that an individual 
is not lost in the system is given by:

\begin{equation*}
    P(L'_1) = \sum_{(u,v) \, \in S_A^{(1)}} \pi(u,v) \hspace{2cm}
    P(L'_2) = \sum_{(u,v) \, \in S_A^{(2)}} \pi(u,v)
\end{equation*}
 
Thus, by using these values as the coefficient of equation 
(\ref{eq:overall_waiting_time_coeff}) the resultant equation can be used to get 
the overall waiting time. 

\begin{equation}\label{eq:overall_waiting_time}
    W = \frac{\lambda_1 P(L'_1)}{\lambda_2 P(L'_2) + \lambda_1 P(L'_1)} W^{(1)} + 
    \frac{\lambda_2 P(L'_2)}{\lambda_2 P(L'_2) + \lambda_1 P(L'_1)} W^{(2)}
\end{equation}
