\subsubsection{Proportion of individuals within target}

Another performance measure that needs to be taken into consideration is the 
proportion of individuals whose waiting and service times lie within a specified 
time target.
In order to consider such measure though one would need to obtain the 
distribution of time in the system for all individuals. 
The complexity of such task lies on the fact that different individuals arrive 
at different states of the Markov model. 
Consider the case when an arrival occurs when the model is at a specific state.

\paragraph{Time distribution at specific state (1 server):}

\begin{figure}[ht]
    \centering
    \scalebox{0.75}{
        \begin{tikzpicture}[-, node distance = 1cm, auto]
\node[state] (u0v0) {(0,0)};
\node[state, right=of u0v0] (u0v1) {(0,1)};
\draw[->](u0v0) edge[bend left] node {\( \Lambda \)} (u0v1);
\draw[->](u0v1) edge[bend left] node {\(\mu \)} (u0v0);
\node[state, right=of u0v1] (u0v2) {(0,2)};
\draw[->](u0v1) edge[bend left] node {\( \Lambda \)} (u0v2);
\draw[->](u0v2) edge[bend left] node {\(\mu \)} (u0v1);
\node[state, below=of u0v2] (u1v2) {(1,2)};
\draw[->](u0v2) edge[bend left] node {\( \lambda_2 \)} (u1v2);
\draw[->](u1v2) edge[bend left] node {\(\mu \)} (u0v2);
\node[state, below=of u1v2] (u2v2) {(2,2)};
\draw[->](u1v2) edge[bend left] node {\( \lambda_2 \)} (u2v2);
\draw[->](u2v2) edge[bend left] node {\(\mu \)} (u1v2);
\node[state, right=of u0v2] (u0v3) {(0,3)};
\draw[->](u0v2) edge[bend left] node {\( \lambda_1 \)} (u0v3);
\draw[->](u0v3) edge[bend left] node {\(\mu \)} (u0v2);
\node[state, right=of u1v2] (u1v3) {(1,3)};
\draw[->](u1v2) edge[bend left] node {\( \lambda_1 \)} (u1v3);
\draw[->](u1v3) edge[bend left] node {\(\mu \)} (u1v2);
\draw[->](u0v3) edge node {\( \lambda_2 \)} (u1v3);
\node[state, right=of u2v2] (u2v3) {(2,3)};
\draw[->](u2v2) edge[bend left] node {\( \lambda_1 \)} (u2v3);
\draw[->](u2v3) edge[bend left] node {\(\mu \)} (u2v2);
\draw[->](u1v3) edge node {\( \lambda_2 \)} (u2v3);
\node[state, right=of u0v3] (u0v4) {(0,4)};
\draw[->](u0v3) edge[bend left] node {\( \lambda_1 \)} (u0v4);
\draw[->](u0v4) edge[bend left] node {\(\mu \)} (u0v3);
\node[state, right=of u1v3] (u1v4) {(1,4)};
\draw[->](u1v3) edge[bend left] node {\( \lambda_1 \)} (u1v4);
\draw[->](u1v4) edge[bend left] node {\(\mu \)} (u1v3);
\draw[->](u0v4) edge node {\( \lambda_2 \)} (u1v4);
\node[state, right=of u2v3] (u2v4) {(2,4)};
\draw[->](u2v3) edge[bend left] node {\( \lambda_1 \)} (u2v4);
\draw[->](u2v4) edge[bend left] node {\(\mu \)} (u2v3);
\draw[->](u1v4) edge node {\( \lambda_2 \)} (u2v4);
\end{tikzpicture}
    }
    \caption{Example Markov model \(C=1, T=2, N=4, M=2\)}
    \label{fig:distribution_of_time_at_specific_state_1_server}
\end{figure}

Consider the Markov model of figure 
\ref{fig:distribution_of_time_at_specific_state_1_server} with one server and a 
threshold of two individuals. 
Assume that an individual of the first type arrives when the model is at state 
\((0,3)\), thus forcing the model to move to state \((0,4)\). 
The distribution of the time needed for the specified individual to exit the 
system from state \((0,4)\) is given by the sum of exponentially distributed 
random variables with the same parameter \(\mu\). 
The sum of such random variables forms an Erlang distribution which is defined 
by the number of random variables that are added and their exponential 
parameter.
Note here that these random variables represent the individual's pathway from 
the perspective of the individual. 
Thus, \(X_i\) represents the time that it takes to move from the 
\(i^{\text{th}}\) position of the queue to the \((i-1)^{\text{th}}\) position 
(i.e. for someone in front of them to finish their service) and \(X_0\) is the 
time it takes to move from having a service to exiting the system.

\begin{align}
    (0,4) \Rightarrow \quad & X_3 \sim Exp(\mu) \nonumber \\
    (0,3) \Rightarrow \quad & X_2 \sim Exp(\mu) \nonumber \\
    (0,2) \Rightarrow \quad & X_1 \sim Exp(\mu) \nonumber \\
    (0,1) \Rightarrow \quad & X_0 \sim Exp(\mu) \nonumber \\
    S = X_3 + X_2 + & X_1 + X_0 = Erlang(4, \mu)
\end{align}

Thus, the waiting and service time of an individual in the model of figure 
\ref{fig:distribution_of_time_at_specific_state_1_server} can be captured by an 
erlang distributed random variable. 
The general CDF of the erlang distribution \(Erlang(k, \mu)\) is given by:

\begin{equation} \label{eq:cdf_erlang}
    P(S < t) = 1 - \sum_{i=0}^{k-1} \frac{1}{i!} e^{-\mu t} (\mu t)^i
\end{equation}

Unfortunately, the erlang distribution can only be used for the sum of 
identically distributed random variables from the exponential distribution. 
Therefore, this approach cannot be used when one of the random variables has a 
different parameter than the others. 
In fact the only case where it can be used is only when the number of servers 
are \(C=1\), or when an individual arrives and goes straight to service 
(i.e. when there is no other individual waiting and there is an empty server).


\paragraph{Time distribution at a state (multiple servers):}

\begin{figure}[h]
    \centering
    \scalebox{0.75}{\begin{tikzpicture}[-, node distance = 1cm, auto]
\node[state] (u0v0) {(0,0)};
\node[state, right=of u0v0] (u0v1) {(0,1)};
\draw[->](u0v0) edge[bend left] node {\( \Lambda \)} (u0v1);
\draw[->](u0v1) edge[bend left] node {\(\mu \)} (u0v0);
\node[state, right=of u0v1] (u0v2) {(0,2)};
\draw[->](u0v1) edge[bend left] node {\( \Lambda \)} (u0v2);
\draw[->](u0v2) edge[bend left] node {\(2\mu \)} (u0v1);
\node[state, below=of u0v2] (u1v2) {(1,2)};
\draw[->](u0v2) edge[bend left] node {\( \lambda_2 \)} (u1v2);
\draw[->](u1v2) edge[bend left] node {\(2\mu \)} (u0v2);
\node[state, below=of u1v2] (u2v2) {(2,2)};
\draw[->](u1v2) edge[bend left] node {\( \lambda_2 \)} (u2v2);
\draw[->](u2v2) edge[bend left] node {\(2\mu \)} (u1v2);
\node[state, right=of u0v2] (u0v3) {(0,3)};
\draw[->](u0v2) edge[bend left] node {\( \lambda_1 \)} (u0v3);
\draw[->](u0v3) edge[bend left] node {\(2\mu \)} (u0v2);
\node[state, right=of u1v2] (u1v3) {(1,3)};
\draw[->](u1v2) edge[bend left] node {\( \lambda_1 \)} (u1v3);
\draw[->](u1v3) edge[bend left] node {\(2\mu \)} (u1v2);
\draw[->](u0v3) edge node {\( \lambda_2 \)} (u1v3);
\node[state, right=of u2v2] (u2v3) {(2,3)};
\draw[->](u2v2) edge[bend left] node {\( \lambda_1 \)} (u2v3);
\draw[->](u2v3) edge[bend left] node {\(2\mu \)} (u2v2);
\draw[->](u1v3) edge node {\( \lambda_2 \)} (u2v3);
\node[state, right=of u0v3] (u0v4) {(0,4)};
\draw[->](u0v3) edge[bend left] node {\( \lambda_1 \)} (u0v4);
\draw[->](u0v4) edge[bend left] node {\(2\mu \)} (u0v3);
\node[state, right=of u1v3] (u1v4) {(1,4)};
\draw[->](u1v3) edge[bend left] node {\( \lambda_1 \)} (u1v4);
\draw[->](u1v4) edge[bend left] node {\(2\mu \)} (u1v3);
\draw[->](u0v4) edge node {\( \lambda_2 \)} (u1v4);
\node[state, right=of u2v3] (u2v4) {(2,4)};
\draw[->](u2v3) edge[bend left] node {\( \lambda_1 \)} (u2v4);
\draw[->](u2v4) edge[bend left] node {\(2\mu \)} (u2v3);
\draw[->](u1v4) edge node {\( \lambda_2 \)} (u2v4);
\end{tikzpicture}}
    \caption{Example Markov model \(C=2, T=2, N=4, M=2\)}
    \label{fig:distribution_of_time_at_specific_state_2_servers}
\end{figure}

Figure \ref{fig:distribution_of_time_at_specific_state_2_servers} represents the 
same Markov model as figure 
\ref{fig:distribution_of_time_at_specific_state_1_server} with the only 
exception that there are 2 servers here. 
By applying the same logic, assuming that an individual arrives at state 
\((0,4)\), the sum of the following random variables arises.

\begin{align}
    (0,4) \Rightarrow \quad & X_2 \sim Exp(2\mu) \nonumber \\
    (0,3) \Rightarrow \quad & X_1 \sim Exp(2\mu) \\
    (0,2) \Rightarrow \quad & X_0 \sim Exp(\mu) \nonumber
\end{align}

Since these exponentially distributed random variables do not share the same 
parameter, an erlang distribution cannot be used. 
In fact, the problem can now be viewed either as the sum of exponentially 
distributed random variables with different parameters or as the sum of 
erlang distributed random variables.
The sum of erlang distributed random variables is said to follow the 
hypoexponential distribution. 
The hypoexponential distribution is defined with two vectors of size equal
to the number of Erlang random variables \cite{Akkouchi2008}, \cite{Smaili2013}. 
The vector \(\vec{r}\) contains all the \(k\)-values of the erlang distributions 
and \(\vec{\lambda}\) is a vector of the distinct parameters as illustrated in
equation (\ref{eq:connection_between_Hypoexponential_Erlang}).

\begin{equation}\label{eq:connection_between_Hypoexponential_Erlang}
    \begin{rcases}
        Erlang(k_1, \lambda_1) \\
        Erlang(k_2, \lambda_2) \\
        \hspace{1cm} \vdots \\
        Erlang(k_n, \lambda_n)
    \end{rcases}
    Hypo(
        \underbrace{(k_1, k_2, \dots k_n)}_{\vec{k}}, 
        \underbrace{(\lambda_1, \lambda_2, \dots \lambda_n)}_{\vec{\lambda}}
    )
\end{equation}

Equivalently, for this particular example:
\begin{align} \label{eq:multiple_servers_distribution_example}
    \begin{rcases}
        \begin{rcases}
            \scriptstyle{X_2 \sim Exp(2\mu)} \\
            \scriptstyle{X_1 \sim Exp(2\mu)}
        \end{rcases}
        \scriptstyle{X_1 + X_2 = S_1 \sim Erlang(2, 2\mu)} \\
        \scriptstyle{X_0 \sim Exp(\mu) \Rightarrow 
        \hspace{1cm} X_0 = S_2 \sim Erlang(1, \mu)}
    \end{rcases}
    \scriptstyle{S_1 + S_2 = H \sim Hypo((2,1), (2\mu, \mu))}
\end{align}

Therefore, the CDF of this distribution can be used to get the probability of 
the time in spent in the system being less than a given target.
The general CDF of the hypoexponential distribution \(Hypo(\vec{r}, 
\vec{\lambda})\), is given by the following expression \cite{Favaro2010}:

\begin{align} \label{eq:general_cdf_hypoexponential}
    & P(H < t) = 1 - \left( \prod_{j=1}^{\mid \vec{r} \mid} \lambda_j^{r_j} \right) 
    \sum_{k=1}^{\mid \vec{r} \mid} \sum_{l=1}^{r_k} \frac{\Psi_{k,l}(-\lambda_k)t^{r_k - l} 
    e^{-\lambda_k t}}
    {(r_k - l)! (l - 1)!} \nonumber \\ 
    & \textbf{where} \qquad \Psi_{k,l}(t) = - \frac{\partial^{l - 1}}
    {\partial t ^{l - 1}} \left( \prod_{j = 0, j \neq k}^{\mid \vec{r} \mid} 
    (\lambda_j + t)^{-r_j} \right) \nonumber \\
    & \textbf{and} \quad \qquad \lambda_0 = 0, r_0 = 1
\end{align}


The computation of the derivative makes equation 
(\ref{eq:general_cdf_hypoexponential}) computationally expensive. 
In \cite{Legros2015} an alternative linear version of that CDF is explored via 
matrix analysis, and is given by the following formula:

\begin{equation} \label{eq:linear_general_cdf_hypoexponential}
    \begin{split}
        F(x) = &1 - \sum_{k=1}^{n} \sum_{l=0}^{k-1} (-1)^{k-1} \binom{n}{k} 
            \binom{k-1}{l} \sum_{j=1}^{n} \sum_{s=1}^{j-1} e^{-x \lambda_s} 
            \prod_{l=1}^{s-1} \left( \frac{\lambda_l}{\lambda_l - \lambda_s} \right)
            ^ {k_s} \\
        & \times \sum_{s < a_1 < \dots < a_{l-1} < j} 
            \left( \frac{\lambda_s}{\lambda_s - \lambda_{a_1}} \right) ^ {k_s}
            \prod_{m=s+1}^{a_1-1} \left( \frac{\lambda_m}{\lambda_m - 
            \lambda_{a_1}}\right) ^ {k_m} \\  
        & \times \prod_{n=a_1}^{a_2-1} \left( \frac{\lambda_n}{\lambda_n - 
            \lambda_{a_2}}\right) ^ {k_n} \dots 
            \prod_{r=a_l-1}^{j-1} \left( \frac{\lambda_r}{\lambda_r - 
            \lambda_{a_j}}\right) ^ {k_r}  
            \sum_{q=0}^{k_s - 1} \frac{((\lambda_s - \lambda_{a_1})x)^q}{q!}, \\
        & \text{for } x \geq 0
    \end{split}
\end{equation}


\paragraph{Specific CDF of hypoexponential distribution}
Equations (\ref{eq:general_cdf_hypoexponential}) and 
(\ref{eq:linear_general_cdf_hypoexponential}) refers to the general CDF of the
hypoexponential distribution where the size of the vector parameters can be of
any size \cite{Favaro2010}.
In the Markov chain models described in figures 
\ref{fig:distribution_of_time_at_specific_state_1_server} and 
\ref{fig:distribution_of_time_at_specific_state_2_servers} the parameter vectors 
of the hypoexponential distribution are of size two, and in fact, for any 
possible version of the investigated Markov chain model the vectors can only be 
of size two.
This is true since for any dimensions of this Markov chain model there will 
always be at most two distinct exponential parameters; the parameter for 
finishing a service (\(\mu\)) and the parameter for moving forward in the queue 
(\(C \mu\)). 
For the special case of \(C=1\) the hypoexponential distribution will not be 
used as this is equivalent to an erlang distribution.
Therefore, by fixing the sizes of \(\vec{r}\) and \(\vec{\lambda}\) to 2, the 
following specific expression for the CDF of the hypoexponential distribution
arises, where the derivative is removed:


\begin{align} \label{eq:specific_cdf_hypoexponential}
    & P(H < t) = 1 - \left( \prod_{j=1}^{\mid \vec{r} \mid} \lambda_j^{r_j} \right) 
    \sum_{k=1}^{\mid \vec{r} \mid} \sum_{l=1}^{r_k} \frac{\Psi_{k,l}(-\lambda_k)t^{r_k - l} 
    e^{-\lambda_k t}}{(r_k - l)! (l - 1)!} \nonumber \\ 
    & \textbf{where} \qquad \Psi_{k,l}(t) = 
    \begin{cases} 
        \frac{(-1)^{l} (l-1)!}{\lambda_2} \left[\frac{1}{t^l} - \frac{1}
        {(t + \lambda_2)^l}\right] , & k=1 \\
        - \frac{1}{t (t + \lambda_1)^{r_1}}, & k=2
    \end{cases} \nonumber \\
    & \textbf{and} \quad \qquad \lambda_0 = 0, r_0 = 1
\end{align}

Note here that the only difference between equations
(\ref{eq:general_cdf_hypoexponential}) and (\ref{eq:specific_cdf_hypoexponential}) 
is the \(\Psi\) function. 
The next part proves that the expression for \(\Psi\) can be simplified for the 
cases of \(k = 1,2\). 
Equation (\ref{eq:hypoexponential_expression_to_proof}) shows the expression to 
be proved.

\begin{equation} \label{eq:hypoexponential_expression_to_proof}
    \Psi_{(k,l)}(t) = - \frac{\partial^{l - 1}}{\partial t ^{l - 1}} 
    \left( \prod_{j = 0, j \neq k}^{\mid \vec{r} \mid} (\lambda_j + t)^{-r_j} \right) = 
    \begin{cases} 
        \frac{(-1)^{l} (l-1)!}{\lambda_2} \left[\frac{1}{t^l} - \frac{1}
        {(t + \lambda_2)^l}\right] , & k=1 \\
        - \frac{1}{t (t + \lambda_1)^{r_1}}, & k=2
    \end{cases}
\end{equation}



\paragraph{Proof of equation (\ref{eq:hypoexponential_expression_to_proof})}
 
This section aims to show that there exists a simplified version of equation 
(\ref{eq:general_cdf_hypoexponential}) that is specific to the proposed Markov 
model.
Function \(\Psi\) is defined using the parameter \(t\) and the variables \(k\) 
and \(l\).
Given the Markov model, the range of values that \(k\) and \(l\) can take can be
bounded.
First of all, from the range of the double summation in equation 
(\ref{eq:general_cdf_hypoexponential}), it can be seen that 
\(k = 1, 2, \dots, \mid \vec{r} \mid\).
Now, \(\mid \vec{r} \mid\) represents the size of the parameter vectors that, 
for the Markov model, will always be 2. 
That is because, for all the exponentially distributed random variables that are
added together to form the new distribution, there only two distinct parameters,
thus forming two erlang distributions. Therefore:

\begin{equation*}
    k = 1, 2
\end{equation*}

By observing equation (\ref{eq:general_cdf_hypoexponential}) once more, the range
of values that \(l\) takes are \(l = 1, 2, \dots, r_k\), where \(r_1\) is 
subject to the individual's position in the queue and \(r_2 = 1\).
In essence, the hypoexponential distribution will be used with these bounds:

\begin{align}
    k = 1 & \qquad \Rightarrow \qquad l = 1, 2, \dots, r_1 \nonumber \\
    k = 2 & \qquad \Rightarrow \qquad l = 1
\end{align}

Thus the left hand side of equation (\ref{eq:hypoexponential_expression_to_proof}) 
needs only to be defined for these bounds. 
The specific hypoexponential distribution investigated here is of the form
\(Hypo((r_1, 1)(\lambda_1, \lambda_2))\).
Note the initial conditions \(\lambda_0=0, r_0=1\) defined in equation 
(\ref{eq:general_cdf_hypoexponential}) also hold here.
Thus the proof is split into two parts, for \(k=1\) and \(k=2\).



\begin{itemize}
    \item \(k = 2, l = 1\)
    \begin{equation*}
        \begin{split}
            LHS &= - \frac{\partial^{1-1}}{\partial t^{1-1}} 
            \left( \prod_{j=0, j \neq 2}^{2} (\lambda_j + t)^{-r_j} \right) \\
            &=-\left( (\lambda_0 + t)^{-r_0} \times (\lambda_1 + t)^{-r_1} \right) \\
            &=-\left( t^{-1} \times (\lambda_1 + t)^{-r_1} \right) \\ 
            &= - \frac{1}{t(t + \lambda_1)^{r_1}} \\
            & \hspace{7cm} \square
        \end{split}
    \end{equation*}
    \item \(k = 1, l = 1, \dots, r_1\)
    \begin{equation*}
        \begin{split}
            LHS &= -\frac{\partial^{l-1}}{\partial t^{l-1}} 
            \left( \prod_{j=0, j \neq 1}^{2} (\lambda_j + t)^{-r_j} \right) \\
            &= -\frac{\partial^{l-1}}{\partial t^{l-1}}
            \left( (\lambda_o + t)^{-r_0} \times (\lambda_2 + t)^{-r_2} \right) \\
            &= -\frac{\partial^{l-1}}{\partial t^{l-1}}
            \left( \frac{1}{t(t + \lambda_2)}\right)
        \end{split}
    \end{equation*}
    In essence, it only remains to show that:
    \[-\frac{\partial^{l-1}}{\partial t^{l-1}} 
    \left( \frac{1}{t(t + \lambda_2)}\right) = \frac{(-1)^{l} (l-1)!}{\lambda_2}
    \left[\frac{1}{t^l} - \frac{1}{(t + \lambda_2)^l}\right]\]
    
    \textbf{Proof by Induction:}
    \begin{enumerate}
        \item Base case (\(l=1\)):
        \begin{equation*}
            \begin{split}
                LHS &= -\frac{\partial^{1-1}}{\partial t^{1-1}} 
                \left( \frac{1}{t(t + \lambda_2)}\right) = 
                - \frac{1}{t(t + \lambda_2)} \\
                RHS &= \frac{(-1)^{1} (1-1)!}{\lambda_2}
                \left[\frac{1}{t^1} - \frac{1}{(t + \lambda_2)^1}\right] \\
                &= - \frac{t + \lambda_2 - t}{\lambda_2 t (t + \lambda_2)} \\
                &= - \frac{1}{t (t + \lambda_2)} \\
                LHS &= RHS
            \end{split}
        \end{equation*}
        \item Assume true for \(l = x\):
        \begin{equation*}
            -\frac{\partial^{x-1}}{\partial t^{x-1}} 
            \left( \frac{1}{t(t + \lambda_2)}\right) = 
            \frac{(-1)^{x} (x-1)!}{\lambda_2}
            \left[\frac{1}{t^x} - \frac{1}{(t + \lambda_2)^x}\right]
        \end{equation*}
        \item Prove true for \(l = x + 1\). Need to show that:
        \[ 
            \frac{\partial^x}{\partial t ^ x} 
            \left( \frac{-1}{t (t + \lambda_2)} \right) = 
            \frac{(-1)^{x + 1} (x)!}{\lambda_2}
            \left[ \frac{1}{t^{x+1}}-\frac{1}{(t + \lambda_2)^{x+1}} \right] 
        \]
        \begin{equation*}
            \begin{split}
                LHS &= \frac{\partial}{\partial t}
                \left[ \frac{\partial^{x-1}}{\partial t ^ {x-1}} 
                \left( \frac{-1}{t (t + \lambda_2)} \right) \right] \\
                &= \frac{\partial}{\partial t} \left[
                    \frac{(-1)^x (x-1)!}{\lambda_2} \left(
                        \frac{1}{t^x} - \frac{1}{(t + \lambda_2)^x}
                    \right)
                \right] \\
                &= \frac{(-1)^x (x-1)!}{\lambda_2} \left(
                    \frac{(-x)}{t^{x+1}} - \frac{(-x)}{(t + \lambda_2)^x}
                \right) \\
                &= \frac{(-1)^x (x-1)! (-x)}{\lambda_2} \left(
                    \frac{1}{t^{x+1}} - \frac{1}{(t + \lambda_2)^x}
                \right) \\
                &= \frac{(-1)^{x+1} (x)!}{\lambda_2} \left(
                    \frac{1}{t^{x+1}} - \frac{1}{(t + \lambda_2)^x}
                \right) \\
                & = RHS \\
                & \hspace{7cm} \square
            \end{split}
        \end{equation*}
    \end{enumerate}
\end{itemize}

\paragraph{Proportion within target for both types of individuals}

Given the two CDFs of the Erlang and Hypoexponential distributions a new 
function has to be defined to decide which one to use among the two.
Based on the state of the model, there can be three scenarios when an individual
arrives.
\begin{enumerate}
    \item There is a free server and the individual does not have to wait
    \begin{equation*}
        X_{(u,v)} \sim Erlang(1, \mu) 
    \end{equation*}
    \item The individual arrives at a queue at the \(n^{th}\) position and the 
    model has \(C > 1\) servers
    \begin{equation*}
        X_{(u,v)} \sim Hypo((n, 1), (C \mu, \mu)) 
    \end{equation*}
    \item The individual arrives at a queue at the \(n^{th}\) position and the 
    model has \(C = 1\) servers
    \begin{equation*}
        X_{(u,v)} \sim Erlang(n + 1, \mu) 
    \end{equation*}
\end{enumerate}

Note here that for the first case \(Erlang(1, \mu)\) is equivalent to 
\(Exp(\mu)\). 
Consider \(X_{(u,v)}^{(1)}\) to be the distribution of type 1 individuals and
\(X_{(u,v)}^{(2)}\) the distribution of type 2 individuals, when arriving at 
state \((u,v)\) of the model.

\begin{equation}
    X_{(u,v)}^{(1)} \sim 
    \begin{cases}
        \textbf{Erlang}(v, \mu), & \textbf{if } C = 1 \textbf{ and } v>1 \\
        \textbf{Hypo} \left(
            \left[v - C, 1\right], \left[C \mu, \mu \right]
        \right), & \textbf{if } C > 1 \textbf{ and } v>C \\
        \textbf{Erlang}(1, \mu), & \textbf{if } v \leq C
    \end{cases}
\end{equation}

\begin{equation}
    X_{(u,v)}^{(2)} \sim 
    \begin{cases}
        \textbf{Erlang}(\min(v, T), \mu), & \textbf{if } C = 1
            \textbf{ and } v, T > 1 \\
        \textbf{Hypo}\left(
            \left[ \min(v, T) - C, 1 \right], \left[ C \mu, \mu \right]
        \right), & \textbf{if } C > 1 \textbf{ and } v, T  > C \\
        \textbf{Erlang}(1, \mu), & \textbf{if } v \leq C \textbf{ or } T \leq C
    \end{cases}
\end{equation}


Thus, the CDF of the random variables \(X_{(u,v)}^{(1)}\) and 
\(X_{(u,v)}^{(2)}\) can be calculated using equations (\ref{eq:cdf_erlang}) and 
(\ref{eq:specific_cdf_hypoexponential}):

\begin{equation}
    P(X_{(u,v)}^{(1)} < t) = 
    \begin{cases}
        1 - \sum_{i=0}^{v-1} \frac{1}{i!} e^{-\mu t} (\mu t)^i, 
            & \textbf{if } C = 1 \\
            & \textbf{and } v>1 \\
        & \\
        1 - (\mu C)^{v-C} \mu  
            \sum_{k=1}^{\mid \vec{r} \mid} \sum_{l=1}^{r_k}
            \frac{\Psi_{k,l}(-\lambda_k)t^{r_k - l} 
            e^{-\lambda_k t}}{(r_k - l)! (l - 1)!},
            & \textbf{if } C > 1 \\
        \textbf{where } \vec{r}=(v - C, 1) \textbf{ and } 
            \vec{\lambda}=(C \mu, \mu) & \textbf{and } v > C \\
        & \\
        1 - e^{-\mu t},  & \textbf{if } v \leq C
    \end{cases}
\end{equation}


\begin{equation}
    P(X_{(u,v)}^{(2)} < t) = 
    \begin{cases}
        1 - \sum_{i=0}^{\min(v,T)-1} \frac{1}{i!} e^{-\mu t} (\mu t)^i,  
            & \textbf{if } C = 1 \\ 
            & \textbf{and } v, T > 1 \\
            & \\
        1 - (\mu C) ^ {\min(v,T) - C} \mu  & \textbf{if } C > 1 \\
        \qquad \times \sum_{k=1}^{\mid \vec{r} \mid} \sum_{l=1}^{r_k} 
            \frac{\Psi_{k,l}(-\lambda_k)t^{r_k - l} 
            e^{-\lambda_k t}}{(r_k - l)! (l - 1)!}, 
            & \textbf{and } v, T  > C \\
        \textbf{where } \vec{r}=(\min(v, T) - C, 1) \\
        \hspace{1.15cm} \vec{\lambda}=(C \mu, \mu) \\
        & \\
        1 - e^{-\mu t}, & \textbf{if } v \leq C \\ 
        & \textbf{or } T \leq C \\
    \end{cases}
\end{equation}


In addition, the set of accepting states for type 1 \(S_A^{(1)}\) and type 2 
\(S_A^{(2)}\) individuals defined in (\ref{eq:accepting_states_type_1}) and 
(\ref{eq:accepting_states_type_2}) are also needed here.
Note here that, \(S\) denotes the set of all states of the Markov chain model. 

\begin{align*}
    S_A^{(1)} &= \{(u, v) \in S \; | \; v < N \} \\
    S_A^{(2)} &=
    \begin{cases}
        \{(u, v) \in S \; | \; u < M \}, & \textbf{if } T \leq N \\
        \{(u, v) \in S \; | \; v < N \}, & \textbf{otherwise}
    \end{cases}
\end{align*}

The following formula uses the state probability vector \(\pi\) to get the 
weighted average of the probability below target of all states in the Markov
model.

\begin{equation}
    P(X^{(1)} < t) = \frac{\sum_{(u,v) \in S_A^{(1)}} P(X_{u,v}^{(1)} < t) 
    \pi_{u,v} }{\sum_{(u,v) \in S_A^{(1)}} \pi_{u,v}}
\end{equation}

\begin{equation}
    P(X^{(2)} < t) = \frac{\sum_{(u,v) \in S_A^{(2)}} P(X_{u,v}^{(2)} < t) 
    \pi_{u,v} }{\sum_{(u,v) \in S_A^{(2)}} \pi_{u,v}}
\end{equation}


\paragraph{Overall proportion within target}

The overall proportion of individuals for both types of individuals is given by 
the equivalent formula of equations (\ref{eq:overall_waiting_time_coeff}) and 
(\ref{eq:overall_waiting_time}).
The following formula uses the probability of lost individuals from both types
to get the weighted sum of the two probabilities.

\begin{equation*}
    P(L'_1) = \sum_{(u,v) \, \in S_A^{(1)}} \pi(u,v), \hspace{1.5cm}
    P(L'_2) = \sum_{(u,v) \, \in S_A^{(2)}} \pi(u,v)
\end{equation*}

\begin{align}\label{eq:overall_proportion_within_target}
    P(X < t) &= \frac{\lambda_1 P(L'_1)}{\lambda_2 P(L'_2)+\lambda_1 P(L'_1)} 
    P(X^{(1)} < t) \\
    &+ \frac{\lambda_2 P(L'_2)}{\lambda_2 P(L'_2) + \lambda_1 
    P(L'_1)} P(X^{(2)} < t)
\end{align}
