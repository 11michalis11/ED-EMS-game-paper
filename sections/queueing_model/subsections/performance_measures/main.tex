\subsection{Performance Measures}


The transition matrix \( Q \) defined in \ref{eq:markov_transition_rate} can be 
used to get the probability vector \( \pi \).
The vector \( \pi \) is commonly used to study stochastic systems and it's main
purpose is to keep track of the probability of being at any given state of 
the system.
The term \textit{steady state} refers to the instance of the vector \( \pi \) 
where the probabilities of being at any state become stable over time. 
Thus, by considering the steady state vector \( \pi \) the relationship between 
it and \( Q \) is given by:

\[
    \frac{d\pi}{dt} = \pi Q = 0
\]

Using vector \(\pi\) there are numerous performance measures of the model that 
can be calculated. 
The following equations utilise \(\pi\) to get performance measures on the 
average number of people at certain sets of state.

\begin{itemize}
    \item Average number of people in the system: 
        \[L = \sum_{i=1}^{|\pi|} \pi_i (u_i + v_i)\]
    \item Average number of people in the service centre: 
        \[L_H = \sum_{i=1}^{|\pi|} \pi_i v_i\]
    \item Average number of people in waiting zone 2:
        \[L_A = \sum_{i=1}^{|\pi|} \pi_i u_i\] 
\end{itemize}

Consequently, there are some additional performance measures of interest that
are not as straightforward to calculate.
Such performance measures are the mean waiting time in the system (for both 
class 1 and class 2 individuals), the mean time blocked in waiting zone 2 (only 
valid for class 2 individuals) and the proportion of individuals that wait in 
waiting zone 1 within a predefined time target.
