\subsection{Performance Measures}


The transition matrix \( Q \) defined in \ref{eq:markov_transition_rate} can be 
used to get the probability vector \( \pi \).
The vector \( \pi \) is commonly used to study stochastic systems and it's main
purpose is to keep track of the probability of being at any given state of 
the system.
The term \textit{steady state} refers to the instance of the vector \( \pi \) 
where the probabilities of being at any state become stable over time. 
Thus, by considering the steady state vector \( \pi \) the relationship between 
it and \( Q \) is given by:

\[
    \frac{d\pi}{dt} = \pi Q = 0
\]

Using vector \(\pi\) there are numerous performance measures of the model that 
can be calculated. 
The following equations utilise \(\pi\) to get performance measures on the 
average number of people at certain sets of state.

\begin{itemize}
    \item Average number of people in the system: 
        \[L = \sum_{i=1}^{|\pi|} \pi_i (u_i + v_i)\]
    \item Average number of people in the service centre: 
        \[L_H = \sum_{i=1}^{|\pi|} \pi_i v_i\]
    \item Average number of people in waiting zone 2:
        \[L_A = \sum_{i=1}^{|\pi|} \pi_i u_i\] 
\end{itemize}

Consequently, there are some additional performance measures of interest that
are not as straightforward to calculate.
Such performance measures are the mean waiting time in the system (for both 
class 1 and class 2 individuals), the mean time blocked in waiting zone 2 (only 
valid for class 2 individuals) and the proportion of individuals that wait in 
waiting zone 1 within a predefined time target.

\subsubsection{Waiting time} \label{sec:waiting_time}

Waiting time is the amount of time that individuals from either class wait in 
waiting zone 1 so that they can receive their service. 
For a given set of parameters there are three different performance measures 
around the mean waiting time that can be calculated; the mean waiting time of
class 1 individuals, the mean waiting time of class 2 individuals and the 
overall mean waiting time. 

Since some of the individuals can be lost to the model, a new set of states 
needs to be defined; the set of \textit{accepting states}. 
That is the set of states that the model is able to accept a certain type of
individual. 
The set of accepting states for class 1 individuals is defined as:

\begin{equation}\label{eq:accepting_states_class_1}
    S_A^{(1)} = \{(u, v) \in S \; | \; v < N \}
\end{equation}

In essence, for class 1 individuals, this is the set of states that are not on 
the last column of states in the Markov chain.
Equivalently, the set of accepting states for class 2 individuals is defined as:

\begin{equation}\label{eq:accepting_states_class_2}
    S_A^{(2)}=
    \begin{cases}
        \{(u, v) \in S \; | \; u < M \}, & \textbf{if } T \leq N\\
        \{(u, v) \in S \; | \; v < N \}, & \textbf{otherwise}
    \end{cases}
\end{equation}

Note here that if the threshold is less than or equal the total capacity of the
system the set includes all states that are not on the last column of the 
Markov chain.
Otherwise, the set of accepting state is identical to 
(\ref{eq:accepting_states_class_1}). Thus, the expressions for the waiting times 
for class 1 and class 2 individuals are given by:

\begin{equation} \label{eq:closed_form_waiting_class_1}
    W^{(1)} = \frac{\sum_{\substack{(u,v) \, \in S_A^{(1)} \\ v \geq C}} 
    \frac{1}{C \mu} \times (v-C+1) \times \pi(u,v)}{\sum_{(u,v) \, 
    \in S_A^{(1)}} \pi(u,v)}
\end{equation}
    
\begin{equation}\label{eq:closed_form_waiting_class_2}
    W^{(2)} = \frac{\sum_{\substack{(u,v) \, \in S_A^{(2)} \\ min(v,T) \geq C}} 
    \frac{1}{C \mu} \times (\min(v+1,T)-C) \times \pi(u,v)}{\sum_{(u,v) \, 
    \in S_A^{(2)}} \pi(u,v)}
\end{equation}

Consequently, the overall waiting time can be estimated by a linear combination 
of \(W_1\) and \(W_2\). 
Thus, the overall waiting time can calculated by the following equation where 
\(c_1\) and \(c_2\) are the coefficients of the terms:

\begin{equation}\label{eq:overall_waiting_time_coeff}
    W = c_1 W^{(1)} + c_2 W^{(2)}
\end{equation}

The two coefficients represent the proportion of individuals of each type that 
did not get lost and traversed through the model.
Thus, one should account for the probability that an individual is lost to the 
system. 
This probability can be easily calculated by using the two sets of accepting 
states \(S_A^{(2)}\) and \(S_A^{(1)}\) defined in equations 
(\ref{eq:accepting_states_class_1}) and (\ref{eq:accepting_states_class_2}). 
Using these equations the probability, for either class type, that an individual 
is not lost in the system is given by:

\begin{equation*}
    P(L'_1) = \sum_{(u,v) \, \in S_A^{(1)}} \pi(u,v) \hspace{2cm}
    P(L'_2) = \sum_{(u,v) \, \in S_A^{(2)}} \pi(u,v)
\end{equation*}
 
Thus, by using these values as the coefficient of equation 
(\ref{eq:overall_waiting_time_coeff}) the resultant equation can be used to get 
the overall waiting time. 

\begin{equation}\label{eq:overall_waiting_time}
    W = \frac{\lambda_1 P(L'_1)}{\lambda_2 P(L'_2) + \lambda_1 P(L'_1)} W^{(1)} + 
    \frac{\lambda_2 P(L'_2)}{\lambda_2 P(L'_2) + \lambda_1 P(L'_1)} W^{(2)}
\end{equation}


\subsubsection{Blocking time}

% TODO: Possibly replace the contents of this section with the ones for the 
% closed form formula of the blocking time
% Currently the direct approach one is shown

Unlike the waiting time, the blocking time is only calculated for individuals of
the second type.  
That is because individuals of the first type cannot be blocked. 
Thus, one only needs to consider the pathway of type 2 individuals to get the 
mean blocking time of the system. 
The set of states where individuals can be blocked is defined as:

\begin{equation} \label{eq:set_of_blocking_states}
    S_b = \{(u,v) \in S \; | \; u > 0\}
\end{equation}
 
In order to not consider individuals that will be lost to the system, the set of 
accepting states needs to be taken into consideration. 
As defined in section \ref{sec:waiting-time}, the set of accepting states is 
given by \ref{eq:accepting-states-class-2}:

\begin{equation*}
    S_A^{(2)}=
    \begin{cases}
        \{(u, v) \in S \; | \; u < M \} & \textbf{if } T \leq N\\
        \{(u, v) \in S \; | \; v < N \} & \textbf{otherwise}
    \end{cases}
\end{equation*}

% For the waiting time formula in sections \ref{sec:recursive-waiting-time-others}
% and \ref{sec:recursive-waiting-time-ambulance}
% the mean sojourn time for each state was considered,
% ignoring any arrivals.
% Here, the same approach is used but ignoring only class 2
% arrivals. That is because for the waiting time formula, once an individual enters 
% the service centre (i.e. starts waiting) any individual arriving after them will 
% not affect their
% pathway. That is not the case for blocking time. When a class 2 individual is 
% blocked, 
% any class 1 individual that arrives will cause the blocked individual to remain 
% blocked for more time. Therefore, class 1 arrivals are considered here:

The mean sojourn time for each state is given by the inverse of the out-flow of
that state.
However, whenever a type 2 individual arrives at the system, no subsequent 
arrival of another type 2 individual can affect its pathway or total time in 
the system.
Therefore, looking at the mean time in the system from the perspective of an 
individual of the second type, all such type 2 arrivals need to be ignored.
Note here that this is not the case for individuals of the first type.
Whenever a type 2 individual is blocked and a type 1 individual arrives the type
2 individuals will remain blocked for some additional amount of time.
Thus, the mean time that a type 2 individual spends at each state is given by:

\begin{equation}\label{eq:time_in_state_blocking_time}
    c(u,v) = 
    \begin{cases}
        \frac{1}{\min(v,C) \mu}, & \text{if } v = N\\
        \frac{1}{\lambda_1 + \min(v,C) \mu}, & \text{otherwise}
    \end{cases}
\end{equation}
 
In equation \ref{eq:time_in_state_blocking_time}, both service completions and 
class 1 arrivals are considered. 
Thus, from a blocked individual's perspective whenever the system moves from one 
state \((u,v)\)
to another state it can either:

\begin{itemize}
    \item be because of a service being completed: we will denote the probability 
    of this happening by \(p_s(u,v)\). 
    \item be because of an arrival of an individual of class 1: denoting such 
    probability by \(p_o(u,v)\).
\end{itemize}
The probabilities are given by:

\begin{equation*}
    p_s(u,v) = \frac{\min(v,C)\mu}{\lambda_1 + \min(v,C)\mu}, \qquad
    p_o(u,v) = \frac{\lambda_1}{\lambda_1 + \min(v,C)\mu}
\end{equation*}


Having defined \(c(u,v)\) and \(S_b\) a formula for the blocking time that is
expected to occur at each state can be given by:

\begin{equation}\label{eq:general_blocking_time_at_each_state}
    b(u,v) = 
    \begin{cases} 
        0, & \textbf{if } (u,v) \notin S_b \\
        c(u,v) + b(u - 1, v), & \textbf{if } v = N = T\\
        c(u,v) + b(u, v-1), & \textbf{if } v = N \neq T \\
        c(u,v) + p_s(u,v) b(u-1, v) + p_o(u,v) b(u, v+1), & \textbf{if } u > 0 
        \textbf{ and } \vspace{-0.2cm} \\ 
        & \quad v = T \\
        c(u,v) + p_s(u,v) b(u, v-1) + p_o(u,v) b(u, v+1), & \textbf{otherwise} \\
    \end{cases}
\end{equation}

A direct approach will be used to solve this equation here. 
By enumerating all equations of (\ref{eq:general_blocking_time_at_each_state}) 
for all states \((u,v)\) that belong in \(S_b\) 
a system of linear equations arises where the unknown variables are all the 
\(b(u,v)\) terms. 
Note here that these equations correspond to all blocking states as defined in
\ref{eq:set_of_blocking_states}. 
Equations that correspond to non-blocking states have a value of \(0\) as 
defined in \ref{eq:general_blocking_time_at_each_state}
The general form of the equation in terms of \(C,T,N \text{ and } M\) is given by: 

\begin{align}
    b(1,T) \quad &= \quad c(1, T) + p_o b(1, T + 1) \label{eq:first_eq_of_blocking_general}\\
    b(1,T + 1) \quad &= \quad c(1, T + 1) + p_s b(1, T) + p_o b(1, T + 1) \\
    b(1,T + 2) \quad &= \quad c(1, T + 2) + p_s b(1, T + 1) + p_o b(1, T + 3) \\
    & \ \, \vdots \nonumber \\
    b(1, N) \quad &= \quad c(1, N) + b(1, N - 1) \\
    b(2, T) \quad &= \quad c(2, T) + p_s b(1, T) + p_o b(2, T + 1) \\
    b(2, T + 1) \quad &= \quad c(2, T + 1) + p_s b(2, T) + p_o b(2, T + 2) \\
    & \ \, \vdots \nonumber \\
    b(M - 1, N) \quad &= \quad c(M, N - 1) + b(M, N-1) \\ 
    b(M, T) \quad &= \quad c(T, N) + p_s b(T-1, N) + p_o b(T, N+1) \\
    & \ \, \vdots \nonumber \\
    b(M, N) \quad &= \quad c(M, N) + b(M, N-1) \label{eq:last_eq_of_blocking_general}
\end{align}

The equivalent matrix notation of the linear system of equations 
(\ref{eq:first_eq_of_blocking_general}) - (\ref{eq:last_eq_of_blocking_general})
is given by \(Zx=y\), where:
\begin{equation}\label{eq:general_algebaric_approach_blocking_time}
    \scalebox{0.73}{\(
        Z = 
        \begin{pmatrix}
            -1 & p_o & 0 & \dots & 0 & 0 & 0 & 0 & 0 & \dots & 0 & 0 \\ %(1,T)
            p_s & -1 & p_o & \dots & 0 & 0 & 0 & 0 & 0 & \dots & 0 & 0 \\ %(1,T+1)
            0 & p_s & -1 & \dots & 0 & 0 & 0 & 0 & 0 & \dots & 0 & 0 \\ %(1,T+2)
            \vdots & \vdots & \vdots & \ddots & \vdots & \vdots & \vdots & 
            \vdots & \vdots & \ddots & \vdots & \vdots \\ 
            0 & 0 & 0 & \dots & 1 & -1 & 0 & 0 & 0 & \dots & 0 & 0 \\ %(1,N)
            p_s & 0 & 0 & \dots & 0 & 0 & -1 & p_o & 0 & \dots & 0 & 0 \\ %(2,T)
            0 & 0 & 0 & \dots & 0 & 0 & p_s & -1 & p_o & \dots & 0 & 0 \\ %(2,T+1)
            \vdots & \vdots & \vdots & \ddots & \vdots & \vdots & \vdots & 
            \vdots & \vdots & \ddots & \vdots & \vdots \\ 
            0 & 0 & 0 & \dots & 0 & 0 & 0 & 0 & 0 & \dots & 1 & -1 \\ %(M,N)
        \end{pmatrix},
        x = 
        \begin{pmatrix}
            b(1,T) \\
            b(1,T+1) \\
            b(1,T+2) \\
            \vdots \\
            b(1,N) \\
            b(2,T) \\
            b(2,T+1) \\
            \vdots \\
            b(M,N) \\
        \end{pmatrix}, 
        y= 
        \begin{pmatrix}
            -c(1,T) \\
            -c(1,T+1) \\
            -c(1,T+2) \\
            \vdots \\
            -c(1,N) \\
            -c(2,T) \\
            -c(2,T+1) \\
            \vdots \\
            -c(M,N) \\
        \end{pmatrix}
    \)}
\end{equation}

Thus, having calculated the mean blocking time for all blocking states \(b(u,v)\), 
it only remains to put them together in a formula.
The resultant formula for the mean blocking time is given by:

\begin{equation}\label{eq:algebraic_blocking_time}
    B = \frac{\sum_{(u,v) \in S_A} \pi_{(u,v)} \; b(u,v)}{\sum_{(u,v) \in S_A} 
    \pi_{(u,v)}}
\end{equation}



To illustrate how the described formula works consider a Markov model where 
\(C=2, T=2, N=4, M=2\) (figure \ref{fig:example-algeb-blocking}). 
The equations that correspond to such a model are shown in 
(\ref{eq:first_eq_of_blocking_example})-(\ref{eq:last_eq_of_blocking_example}) 
and their equivalent matrix notation form is shown in 
\ref{eq:example_algebaric_approach_blocking_time}.

\begin{minipage}{.5\textwidth}
    \begin{figure}[H]
        \scalebox{0.6}{\begin{tikzpicture}[-, node distance = 1cm, auto]
\node[state] (u0v0) {(0,0)};
\node[state, right=of u0v0] (u0v1) {(0,1)};
\draw[->](u0v0) edge[bend left] node {\( \Lambda \)} (u0v1);
\draw[->](u0v1) edge[bend left] node {\(\mu \)} (u0v0);
\node[state, right=of u0v1] (u0v2) {(0,2)};
\draw[->](u0v1) edge[bend left] node {\( \Lambda \)} (u0v2);
\draw[->](u0v2) edge[bend left] node {\(2\mu \)} (u0v1);
\node[state, below=of u0v2] (u1v2) {(1,2)};
\draw[->](u0v2) edge[bend left] node {\( \lambda_2 \)} (u1v2);
\draw[->](u1v2) edge[bend left] node {\(2\mu \)} (u0v2);
\node[state, below=of u1v2] (u2v2) {(2,2)};
\draw[->](u1v2) edge[bend left] node {\( \lambda_2 \)} (u2v2);
\draw[->](u2v2) edge[bend left] node {\(2\mu \)} (u1v2);
\node[state, right=of u0v2] (u0v3) {(0,3)};
\draw[->](u0v2) edge[bend left] node {\( \lambda_1 \)} (u0v3);
\draw[->](u0v3) edge[bend left] node {\(2\mu \)} (u0v2);
\node[state, right=of u1v2] (u1v3) {(1,3)};
\draw[->](u1v2) edge[bend left] node {\( \lambda_1 \)} (u1v3);
\draw[->](u1v3) edge[bend left] node {\(2\mu \)} (u1v2);
\draw[->](u0v3) edge node {\( \lambda_2 \)} (u1v3);
\node[state, right=of u2v2] (u2v3) {(2,3)};
\draw[->](u2v2) edge[bend left] node {\( \lambda_1 \)} (u2v3);
\draw[->](u2v3) edge[bend left] node {\(2\mu \)} (u2v2);
\draw[->](u1v3) edge node {\( \lambda_2 \)} (u2v3);
\node[state, right=of u0v3] (u0v4) {(0,4)};
\draw[->](u0v3) edge[bend left] node {\( \lambda_1 \)} (u0v4);
\draw[->](u0v4) edge[bend left] node {\(2\mu \)} (u0v3);
\node[state, right=of u1v3] (u1v4) {(1,4)};
\draw[->](u1v3) edge[bend left] node {\( \lambda_1 \)} (u1v4);
\draw[->](u1v4) edge[bend left] node {\(2\mu \)} (u1v3);
\draw[->](u0v4) edge node {\( \lambda_2 \)} (u1v4);
\node[state, right=of u2v3] (u2v4) {(2,4)};
\draw[->](u2v3) edge[bend left] node {\( \lambda_1 \)} (u2v4);
\draw[->](u2v4) edge[bend left] node {\(2\mu \)} (u2v3);
\draw[->](u1v4) edge node {\( \lambda_2 \)} (u2v4);
\end{tikzpicture}}
        \caption{
            \centering{Example of Markov chain with \(C=2, T=2, N=4, M=2\)}
        }
        \label{fig:example-algeb-blocking}
    \end{figure}
\end{minipage}
\begin{minipage}{.43\textwidth}
    \begin{align}
        b(1,2) &= c(1,2) + p_o b(1,3) \label{eq:first_eq_of_blocking_example} \\
        b(1,3) &= c(1,3) + p_s b(1,2) \nonumber \\ &+ p_o b(1,4) \\
        b(1,4) &= c(1,4) + b(1,3) \\
        b(2,2) &= c(2,2) + p_s b(1,2) \nonumber \\ &+ p_o b(2,3) \\
        b(2,3) &= c(2,3) + p_s b(2,2) \nonumber \\ &+ p_o b(1,4) \\
        b(2,4) &= c(2,4) + b(2,3) \label{eq:last_eq_of_blocking_example}
    \end{align}
\end{minipage}

\begin{equation}\label{eq:example_algebaric_approach_blocking_time}
    Z=
    \begin{pmatrix}
        -1 & p_o & 0 & 0 & 0 & 0 \\ %(1,2)
        p_s & -1 & p_o & 0 & 0 & 0 \\ %(1,3)
        0 & 1 & -1 & 0 & 0 & 0 \\ %(1,4)
        p_s & 0 & 0 & -1 & p_o & 0\\ %(2,2)
        0 & 0 & 0 & p_s & -1 & p_o \\ %(2,3)
        0 & 0 & 0 & 0 & 1 & -1 \\ %(2,4)
    \end{pmatrix},
    x=
    \begin{pmatrix}
        b(1,2) \\
        b(1,3) \\
        b(1,4) \\
        b(2,2) \\
        b(2,3) \\
        b(2,4) \\
    \end{pmatrix}, 
    y=
    \begin{pmatrix}
        -c(1,2) \\
        -c(1,3) \\
        -c(1,4) \\
        -c(2,2) \\
        -c(2,3) \\
        -c(2,4) \\
    \end{pmatrix}
\end{equation}
