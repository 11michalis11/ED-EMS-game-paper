\section{Discrete Event Simulation}\label{sec:appendix_des}

For the purposes of this study, a discrete event simulation (DES) model was
constructed to support the Markov chain version described in 
section~\ref{sec:queueing_model}.
The queueing model was built in python using the Ciw 
library~\cite{ciwpython}.

The constructed model simulates a queueing system with two waiting spaces and 
two types of individuals. 
The expected behaviour of the nodes in Ciw have been modified such that 
individuals moving from waiting zone 2 into waiting zone 1 get blocked 
if there are more than \(T\) individuals in waiting zone 1.

The same performance measures described in sections 
\ref{sec:waiting_time}, \ref{sec:blocking_time} and 
\ref{sec:proportion_within_target} can also be calculated using the DES model.
The simulation can be ran a number of times to eliminate stochasticity and the
outcomes of the two methods can be directly comparable. 

This discrete event simulation model was built in python and can be installed by 
running the following command in the terminal:
\begin{lstlisting}[language=bash, style=terminalstyle]
    $ pip install ambulance_game
\end{lstlisting}

Having installed the package, the following script can be used to simulate, and 
get all required performance measures, of a queueing system with two waiting 
spaces and two types of individuals for a runtime of 2000 unit time with the 
following parameters:

\begin{multicols}{2}
    \begin{itemize}
        \centering
        \item \( \lambda_1 = 2 \)
        \item \( \lambda_2 = 3 \)
        \item \( \mu = 1 \)
        \item \( C = 6 \)
        \columnbreak
        \item \( T = 10 \)
        \item \( N = 20 \)
        \item \( M = 10 \)
        \item \( R = 1 \)
    \end{itemize}
\end{multicols}

\lstinputlisting[language=Python, style=pystyle]{sections/appendix/descrete_event_simulation/simulation.py}

The script above shows the outcome of the DES model for a single run.
To minimise the stochasticity of the simulation, the model can be simulated for 
numerous trials. 
The following script can be used to run the simulation for a number of times 
and get the mean of the waiting, service and blocking times:

\lstinputlisting[language=Python, style=pystyle]{sections/appendix/descrete_event_simulation/multiple_runs.py}


Finally, the proportion of individuals within target for multiple runs can be
calculated by running the following block of code:

\lstinputlisting[language=Python, style=pystyle]{sections/appendix/descrete_event_simulation/multiple_runs_proportion.py}

% TODO: Archive simulation python code
